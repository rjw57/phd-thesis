\begin{savequote}
\quoteperson{They can't keep this level of graphics up for much longer!
We used to be lucky if we only got three shades of gray, let alone any 
real colours!}{Cranky Kong}
\end{savequote}

\chapter{Hardware Assisted Geometric Algebra on the GPU}

In this chapter we explore how the \emph{Graphics Processing
Units} (GPUs) in modern consumer-level graphics cards can be
used to perform Geometric Algebra computations faster than a 
typical single-core CPU.

\section{An Overview of GPU Architecture}

\begin{figure}
\centering
\includegraphics[width=0.9\textwidth]{gpu_architecture}
\caption{\label{fig:gpu_architecture}%
  A simplified block diagram of a typical GPU.}
\end{figure}

The GPU in a graphics card is not designed for general purpose 
computing. It is designed, not suprisingly, to perform the
sort of operations useful for graphics rendering. Figure
\ref{fig:gpu_architecture} shows a simplified block diagram of
a typical GPU.

In a traditional fixed-function GPU the CPU uploads, over the AGP
bus, a set of vertices, texture maps and some state information. The
state includes projection matrices, view matrices, clipping planes, 
lighting models, etc. The vertex list and texture maps (if present) 
are then stored in some on-card memory.

On the card there exists a number of rendering pipelines, each of which
can be run in parallel to increase throughput. The pipeline consists
of a \emph{vertex shader} which fetches triplets of vertices from
the vertex memory, transforms them using the current projection and 
view matrices, performs any clipping and sends them to the rasterizer.

The rasteriser takes triplets of screen co-ordinate vertices, forms
a triangle from them and outputs a set of pixel positions which form
render the triangle along with depth information and interpolated texture
co-ordinates. 

The fragment shader takes these pixel positions and, using the texture maps
stored in texture memory along with appropriate state information calculates
the colour of the pixel after all lighting, etc. is performed. The shaded pixel
is then sent to the screen.

In reality the rendering stage is split at the rasterizer allowing for differing
numbers of vertex and fragment shaders but for the purposes of GPU programming
one can view the shaders as being within the same pipeline.

In modern programmable GPUs both the vertex and fragment shaders are fully 
programmable allowing different per-vertex and per-pixel transformation
and shading that is allowed in the traditional fixed-function pipeline.

Each shader is, in effect, an efficient vector processor and, on modern 
graphics cards, there are a number working in parallel. To perform
general purpose computation on the GPU we must find ways of modifying our
algorithms to fit this model.

\section{GPU Programming Methods}

\subsection{DirectX shader language}

Microsoft's DirectX\cite{GPU:DirectX} is a graphics programming API for
Windows and, to some extent, the Microsoft XBox gaming platform. As part of
the API it specifies a generic shading language\cite{GPU:DirectXShadingLanguage}
which abstracts the vertex and fragment shaders. Each version of DirectX
specifies a minimum set shader capabilities which \emph{must} be supported by a 
card claiming compatibility with that level of the API. Consequently a shader written in
DirectX's shading language is portable across all cards which support that level
of the API. A disadvantage of DirectX is that it doesn't expose any functionaility
beyond the specified minimum and it is not portable across operating systems.

\subsection{OpenGL shader language}

OpenGL\cite{GPU:OpenGLSpec} is a cross-platform C-based graphics API. 
It is the \emph{de facto} standard API for non-Windows platforms and
is well supported on Windows platforms by both ATI and nVidia, the dominant
vendor at the time of writing. 

The OpenGL 2.0 specification\cite{GPU:OpenGL2Overview} proposed by 3DLabs
includes a hardware agnostic shading language\cite{GPU:OpenGLShadingLanguage}
known as \emph{GL Shading Language} (GLSL) which provides a similar level
of functionality to that exposed in DirectX. 

Currently few hardware vendors support OpenGL 2.0 --- at the time of writing
nVidia was the only mainstream vendor to provide support in their
consumer-level hardware\cite{nvidia:7664relnotes}.

OpenGL 1.4 and below provide support for programmable shaders via a set of per-vendor
extensions. Unfortunately these requires progamming the GPU in a variety of
assembly languages and vary between different GPUs.

\subsection{The Cg toolkit from nVidia}

Both the OpenGL and DirectX shading languages have a number of disadvantages
from the point of view of research. The OpenGL shading language, whilst being
high-level and convenient to program in, is not widely available. The
DirectX toolkit porvides only a `lowest common denomitator' shading language,
it is not particularly high-level and is, for all intents and purposes, limited
to the Microsoft Windows operating systems. 

An attempt to bridge these gaps is the Cg toolkit\cite{nvidia:cgtoolkit} from nVidia.
It provides a high-level C-like shading language which may be compiled at run-time or
ahead of time into either DirectX shading language or the myrriad of OpenGL extensions
which exist for accessing the programmable shaders. 

Along with these advantages is the `future-proof' nature of Cg. The core Cg language
is easily extended via so-called `profiles'. These either restrict the language to
correspond to the capabilities of particular GPUs or add more features to expose
new GPU abilities. As an example in figure \ref{fig:gpu_architecture} it was implied
that the vertex shader cannot access the texture-map memory. This was true
of older cards but the latest generation of nVidia cards can
access texture maps from within the vertex shader\cite{nvidia:sm3unleashed}. Addid this
capability to Cg was simply a case of releasing a new profile where texture-map related
parts of the language were now available from within both shaders. Older shaders
would still work however and the presence of this capability could be queried
at run-time.

Because of these advantages it was decided to use Cg as the basis for the GPU
computing research. None of the techniques described below require it as they
can all be re-structured to use alternate APIs.

\section{Mesh deformation}

In this section we shall give an example of a vertex shader that uses GA to
perform mesh deformation.

\section{Cloth simulation}

In this section we shall give an example of a fragment shader which is used to
perform a simple cloth simulation.
