%\begin{savequote}
%\quoteperson{A picture is worth 10K words --- but only those 
%  to describe the picture. Hardly any sets of 10K words can
%          be adequately described with pictures.}{Alan Perlis}
%\end{savequote}

%\chapter{Visualisation}

%In this chapter we will discuss some methods for visualising Euclidean objects
%in CGA as implemented by {\tt libcgadraw}, a companion library to {\tt libcga}.

%% AUTHOR NOTE
% This was originally a separate chapter. I didn't like that.

\section{Visualising Objects within the Algebra}

Many CGA visualisation solutions exist\cite{CLU}. The {\tt libcgadraw}
companion library to {\tt libcga} was created as a convenience library
for visualising the output of algorithms implemented with
{\tt libcga}. The library
uses an approach similar to that of the CLUDraw library were by the results of 
a GA-based algorithm may be displayed directly on the screen.
The output of GA-based algorithms may be categorised into a number of
geometric primitives they represent. Below is an a selection of the
visualisation algorithms used. In section \ref{sec:noneuclid-vis} we will
develop some further algorithms for visualising the output of algorithms
working in non Euclidean geometries.

\subsection{Point Pairs}

Point pairs are represented by the wedge product of the representations of the
points. That is to say that the point pair $(a,b)$ is represented by $F(a) \wedge F(b)$.
Using the method of projectors in section \ref{sec:projectors} the representations
of $a$ and $b$ can be extracted and plotted as shown in algorithm \ref{alg:pointpair}.

\begin{fancyalg}
\begin{algorithmic}[1]
\REQUIRE{$T$, a representation of a point-pair.}
\STATE $T' := \frac{1}{(T \cdot n)^2} T$
\STATE $P := \frac{1}{2}(1+T')$
\STATE $B := P \left[ T\cdot n \right]$
\STATE $A := - \tilde{P} \left[ T \cdot n \right]$
\STATE $a := F^{-1}(A)$
\STATE $b := F^{-1}(B)$
\STATE glBegin(GL\_POINTS);
\STATE glVertex($a$); glVertex($b$);
\STATE glEnd();
\end{algorithmic}
\caption{\label{alg:pointpair}Extracting and rendering a point pair.}
\end{fancyalg}

\subsection{Lines}

We render lines by intersecting them with some sphere centred on the origin
and drawing a line between the two points of intersection. The radius of the
sphere used effectively provides a upper limit to the distance a line may be
from the origin while still being rendered. If the line is further from the 
origin than the radius of the sphere it will not intersect the sphere.

This is in actual fact a desirable property since it allows for a 'horizon' to
be set. Lines, being infinite in extent, have end points at infinity which
cannot easily be expressed within OpenGL. Instead by setting the radius of the sphere
to be sufficiently large we may approximate such lines sufficiently for rendering
in many problems. The specific method is given in algorithm \ref{alg:line}.

\begin{fancyalg}
\begin{algorithmic}[1]
\REQUIRE{$L$, a representation of a line.}
\REQUIRE{$r_{\mathrm{max}}$, maximum distance from origin to render lines.}
\STATE $r := 1$
\REPEAT
\STATE intersect := false
\STATE $S := $ representation of sphere centred on origin radius $r$.
\STATE $T := L \vee S$
\IF{$T \neq 0$}
\STATE intersect := true
\STATE Extract point pair from $T$ to $a,b$ as in algorithm \ref{alg:pointpair}.
\ENDIF
\STATE $r := r+1$
\UNTIL{intersect = true {\bfseries or} $r > r_{\mathrm{max}}$}
\IF{intersect = true}
\STATE glBegin(GL\_LINES);
\STATE glVertex($a$); glVertex($b$);
\STATE glEnd();
\ENDIF
\end{algorithmic}
\caption{\label{alg:line}Rendering the representation of a line, $L$.}
\end{fancyalg}

%\subsection{Planes}
%
%A plane, $P$ is represented by three points, $P_1, P_2$ and $P_3$, as
%\[
%P = P_1 \wedge P_2 \wedge P_3 \wedge n.
%\]

\subsection{Planes}

A plane may easily be rendered via its dual representation\cite{wareham_lasenby}.
If $p = PI \equiv P^*$ then
\[
p = \hat{n} + dn
\]
where $\hat{n}$ is the plane normal and $d$ is its perpendicular distance from the origin.

\subsection{Circles}

Recall that the circle passing through the point-representations $P_1, P_2$ and $P_3$
is represented by $C = P_1 \wedge P_2 \wedge P_3$ and the plane common to these points is
$P_1 \wedge P_2 \wedge P_3 \wedge n$ hence we can easily find the plane of 
the circle, $P$ via
\[
P = C \wedge n
\]

To find the radius and centre of the circle we use the alternate form of the 
circle
\cite{wareham_lasenby}
\[
C^* = B - \frac{1}{2}\rho^2n
\]
where $C^* \equiv CI_4$ here (the dual of $C$ with respect to the pseudoscalar 
for the plane), $B$ is the centre of the circle and $\rho$ is the radius. The
radius can be found simply using
\begin{eqnarray*}
(C^*)^2 & = & \left(B - \frac{1}{2}\rho^2n\right)\left(B - \frac{1}{2}\rho^2n\right) \\
        & = & -\rho^2 B\cdot n = \rho^2
\end{eqnarray*}
using $B^2 = n^2 = 0$ and $B \cdot n = -1$. Similarly we can find $B$ via
\[
B = C^* \left( 1 + \frac{1}{2}C^*n \right)
\]
The final algorithm (for three dimensions) is outlined in algorithm \ref{alg:circle}.
In this algorithm the function spatial$()$ extracts the components of the argument
which contain no part of $e$ or $\bar{e}$.

\begin{fancyalg}
\begin{algorithmic}[1]
\REQUIRE{$C$, a representation of a circle.}
\STATE $\rho := \sqrt{(C^*)^2}$
\STATE $b := F^{-1}\left(C^* \left[ 1 + \frac{1}{2}C^*n \right]\right)$
\STATE $P := C \wedge n$
\STATE $\hat{n} := {\rm spatial}(P^*)$
\STATS ${\rm Draw circle in plane with normal~}\hat{n}{\rm, centre~}
b{\rm, radius~}\rho.$
\end{algorithmic}
\caption{\label{alg:circle}Rendering the representation of a circle, $C$.}
\end{fancyalg}

\subsection{Spheres}

Finally spheres may be rendered simply by noting that circles are two-dimensional
spheres and all the formul\ae for obtaining centres and radii hold true. Extracting
the null-vector representation of the centre, $B = F(b)$, and the radius, $\rho$,
of a particular sphere $\Sigma$ is therefore straightforward:
\[
B = \Sigma n \Sigma
\]
and
\[
\rho^2 = \left(\Sigma^*\right)^2.
\]
