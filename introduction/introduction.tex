\begin{savequote}
\quoteperson{You know we all became mathematicians for the same reason: 
  we were lazy.}{Max Rosenlicht}
\end{savequote}

\chapter{Introduction}

It has long been known that going to a 4d, projective, description
of 3d Euclidean space can have various advantages -- particularly
when intersections of planes and lines are required. Such
projective descriptions are used extensively in computer vision
where rotations and translations can be described by a single
$4\times 4$ matrix and non-linear projective transformations
become linear. This 4d description of projective geometry fits
very nicely into the Geometric Algebra framework and applications
are given in \cite{HZ91,IJPR00}. 
Since its inception in the mid-1970s, Computer Graphics (CG) has almost
universally used linear vector algebra as its mathematical framework.
This is due primarily to two factors; most early practitioners of
computer graphics were mathematicians familiar with it and 
linear algebra provided a compact, efficient way of representing points, 
transformations, lines, etc.

In the early-1980s CG moved out of the realm of Computer
Science research and started to be used in the broader scientific 
community as an important research tool, both for simulation and visualisation.
Computer Graphics was still tied to classical vector algebra which
had started to show a number of problems when applied to the problems
being investigated. Amongst these were poor generalisation to
spaces other than $\mathbb{R}^3$ and great conceptual difficulty 
in extending problems to non-Euclidean spaces and
manipulating geometric objects other than simple lines, planes and points.

As computing power becomes cheaper, the opportunity
arises to investigate new
frameworks for CG which, although not providing the time/space
efficiency of vector algebra, may provide a conceptually simpler
system or one of greater analytical power.

\section{A Brief Overview of Geometric Algebra}

As stated above, classical vector algebra has a number of problems
once one moves away from three-dimensional Euclidean space. Perhaps the
clearest example is the cross-product of two vectors.  The product, $\mathbf{a}
\times \mathbf{b}$, is conventionally defined as a vector normal to the plane
that $\mathbf{a}$ and $\mathbf{b}$ both lie in and has magnitude $ab\sin\theta$
where $\theta$ is the angle between $\mathbf{a}$ and $\mathbf{b}$. However, the
normal to the plane is only uniquely defined in 3-dimensions and has no meaning
in 2- or 1-dimensional space; the cross-product does not generalise to
higher-dimension spaces. The product is an important element of vector
algebra and one can see that performing geometric operations in higher-spaces
without it quickly becomes complex.

\subsection{The products}

It was in an attempt \cite{GA:grassmann} to create an algebra of vectors which 
did generalise
to higher spaces that a German schoolteacher named Hermann 
Gra{\ss}mann (1809--1877) created an \emph{exterior} 
or \emph{outer} product of two
vectors denoted as $a \wedge b$. 
%Note that we have dropped the normal
%typographic convention of denoting vectors in boldface for
%reasons which will become clear later.

Gra{\ss}mann's outer product can be visualised geometrically as moving
one vector along the other to form a `directed area'. This is a new object,
neither a vector or a scalar. We shall refer to it as a \emph{bivector}.
Similarly one may form the outer product of this bivector with another
vector to form a directed volume or \emph{trivector}. We say a 
bivector has grade 2, a trivector has grade 3 and, generally, a
$n$-vector has grade $n$.

\begin{figure}
\centering
\includegraphics{geometric}
\caption{Illustration of bi- and trivectors\label{fig:geometric}}
\end{figure}

The geometric objects are illustrated in Figure \ref{fig:geometric}. A 
key feature is that the outer-product is anti-commutative and 
associative so that
\begin{displaymath}
a \wedge b = - ( b \wedge a)\quad\mbox{and}\quad 
a \wedge (b \wedge c) = (a \wedge b) \wedge c = a \wedge b \wedge c
\end{displaymath}

Most of Gra{\ss}mann's work was largely ignored by the mathematical community.
It was not until William Clifford (1845--1879) investigated Gra{\ss}mann's
algebra in 1878 \cite{GA:clifford} that the crucial step was made.

Clifford unified Gra{\ss}mann's outer product and the familiar dot
or inner-product into one geometric product such that
\begin{displaymath}
ab = a\cdot b + a \wedge b
\end{displaymath}

A second glance at this shows an interesting feature which should be noted.
In the expression above we are adding a scalar ($a \cdot b$) to a 
bivector ($a \wedge b$). This combination of different objects is
analogous to complex numbers where we linearly combine
a real and imaginary number to form a complex number. In this case we
refer to a linear combination of objects of varying grade as a
\emph{multivector}. Since the algebra may be extended to deal with
multivectors we no longer need different notation for scalars, vectors,
bivectors, etc. This project will use the convention of using lower-case
letters to refer to single-grade objects and upper-case letters to refer
to multivectors. The geometric product also gives us a convenient new
definition of the outer and inner products for vectors
\[
A \wedge B = \frac{1}{2}(AB - BA)
\]
\[
A \cdot B = \frac{1}{2}(AB + BA)
\]

The power of this approach may be illustrated through its application
to rotation. In 2-D this is easily performed using complex numbers. Representing
the vector $[x\ y]$ as the complex number $z = x + iy$, rotation by
$\theta$ radians can be performed by multiplication with $e^{i\theta}$.
William Hamilton (1805--1865) worked for many years to extend this approach
to 3-D. He eventually created \emph{quaternions} \cite{hamilton2,hamilton1},
an algebra with 4 
elements $\left\{1, \mathbf{i}, \mathbf{j}, \mathbf{k}\right\}$ which, although
functional, lacked an obvious geometrical interpretation and again didn't
generalise easily to higher-dimensions.

\subsection{Rotation via Rotors}

To demonstrate Clifford's approach, consider three orthonormal basis
vectors of $\mathbb{R}^3$, $\left\{e_1, e_2, e_3\right\}$. We can form
3 different bivectors from these vectors:
\begin{displaymath}
B_1 = e_2e_3,\quad B_2 = e_3e_1,\quad B_3 = e_1e_2
\end{displaymath}

Note that these are indeed bivectors since
\[
e_ie_j = e_i \cdot e_j + e_i \wedge e_j = e_i \wedge e_j \quad \mbox{iff} \quad i \ne j
\]

Now consider the effect of $B_3$ on the vectors $e_1$ and $e_1 + e_2$:
\begin{displaymath}
e_1B_3 = e_1e_1e_2 = e_1^2e_2=e_2 
\end{displaymath}
\begin{displaymath}
(e_1 + e_2)B_3 = e_1B_3 + e_2B_3 = e_2 + e_2e_1e_2 = e_2 - e_1e_2^2 = e_2 - e_1
\end{displaymath}

\begin{figure}
\centering
\includegraphics{rotation}
\caption{The rotation effect of bivector $B_3 = e_1e_2$\label{fig:rotation}}
\end{figure}

By looking at Figure \ref{fig:rotation} it is clear that $B_3$ has the
effect of rotating the vectors counter-clockwise by 90 degrees. It is, in
fact, a general property that the bivector $e_ie_j$ will rotate a
vector 90 degrees in the plane defined by $e_i$ and $e_j$. At first 
glance this seems to offer little more than quaternions but at no point 
have we assumed that we are working in 3-dimensional space; this method
also works in higher-dimension spaces.

Now we extend to general rotations. Firstly it is trivial to
show that $B_3$ squares to $-1$:
\begin{displaymath}
B_3^2 = e_1e_2e_1e_2 = -e_1e_2e_2e_1 = -1
\end{displaymath}

We can represent any vector $Z$ in the plane defined by $e_1$ and
$e_2$ using
\begin{eqnarray*}
Z & = & r ( e_1 \cos \theta + e_2 \sin \theta) \\
  & = & e_1 r ( \cos \theta + B_3 \sin \theta)
\end{eqnarray*}
where $r$ is the distance of the point $Z$ from the origin
and $\theta$ is the angle $Z$ makes to $e_1$. Also note
that $e_1B_3 = e_1^2e_2 = e_2$. By taking the Taylor expansion of cosine
and sine and re-arranging the coefficients it can be shown that
\[
e^{B_3\theta} = \cos \theta + B_3 \sin \theta
\]
which is the analogous form of de Moivre's theorem for complex
numbers.

We can thus represent any vector $Z$ which lies in the plane of the
bivector $B_3$ by
\[
Z=e_1re^{B_3\theta}
\]

From this the same argument used for rotation in the complex plane 
can be used to show that rotation by
$\phi$ radians in the plane of $B_3$ is accomplished by $Z \mapsto Z'$
where
\begin{displaymath}
Z' = Ze^{B_3\phi} = Z (\cos \phi + B_3 \sin \phi)
\end{displaymath}

This has all taken place in two dimensions for the moment but note that nothing
in our development has assumed this. In fact we can define a bivector $B_3 = ab$ in
three dimensions and rotate vectors in the place defined by $a$ and $b$ using
the expression above.

\begin{figure}
\centering
\includegraphics{rotation2}
\caption{Rotating vectors in arbitrary planes\label{fig:rotation2}}
\end{figure}

Careful consideration must be given to the case where the vector to be
rotated, $x$, does not lie on the plane of rotation as in Figure
\ref{fig:rotation2}. Firstly decompose the
vector into a component which lies in the plane $x_\parallel$ and one
normal to the plane $x_\perp$
\[
x = x_\parallel + x_\perp
\]

Now consider the effect of the following
\begin{eqnarray*}
e^{-B_3\phi/2}
x
e^{B_3\phi/2}
& = & \left(\cos \frac{\phi}{2} - B_3 \sin \frac{\phi}{2}\right)
(x_\parallel + x_\perp )
\left(\cos \frac{\phi}{2} + B_3 \sin \frac{\phi}{2}\right) \\
& = & x_\parallel (\cos \phi + B_3 \sin \phi) + x_\perp\\ 
\end{eqnarray*}
since bivectors anti-commute with vectors in their plane (e.g. 
$e_1(e_2e_1) = -e_2 = -(e_2e_1)e_1$) and commute with
vectors normal to the plane (e.g.\ $e_1(e_2e_3) = (e_2e_3)e_1$).
We have thus succeeded in rotating the component of the vector 
which lies in the plane without affecting the component normal
to the plane --- we have rotated the vector around an axis normal to
the plane.

%and hence
%\begin{eqnarray*}
%Z (\cos \phi + B_3 \sin \phi) & = & (\cos \phi - B_3 \sin \phi)Z \\
%& = &
%\left(\cos \frac{\phi}{2} - B_3 \sin \frac{\phi}{2}\right)
%Z
%\left(\cos \frac{\phi}{2} + B_3 \sin \frac{\phi}{2}\right)
%\end{eqnarray*}

This leads to a general method of rotation in any plane; we
form a bivector of the form $R = \exp{-B\phi/2}$ for a given rotation 
$\phi$ in a plane specified by the bivector $B$ and the transformation
is given by
\begin{displaymath}
x \mapsto RxR^{-1}
\end{displaymath}
We refer to these bivectors which have a rotational effect as \emph{rotors}.
Figure \ref{fig:rotation2} shows the various objects used.

The computation of $R^{-1}$ is rather difficult analytically (and indeed can
require a full $2^n$-dimension matrix inversion for a space of
dimension $n$). To combat this we define
the \emph{reversion} of a $n$-vector $X = e_ie_j...e_k$ as
\[
\tilde{X} = e_k...e_je_i
\]
i.e. the literal reversion of the components. By looking at the expression for
$R$ it is clear that $\tilde{R} \equiv R^{-1}$ for rotors. Computing $\tilde{R}$
is easier since it is simply a permutation.

Note that in spaces with dimension $n$ the maximum grade object 
possible is an $n$-grade
one. We denote the $n$-vector $e_1 \wedge ... \wedge e_n = I$ as the \emph{pseudoscalar}
and the product $xI$ as the \emph{dual} of a vector or $x^*$. The dual is also defined
for general multivectors.

\subsection{The Conformal Model}

In the Conformal Model \cite{GA:NFCM} we extend the space by adding two 
additional basis
vectors. 
The notation we use will follow the original notation given in
\cite{HS84}. Let $x$ be a vector in a space denoted ${\cala}(p,q)$.
The annotation $(p,q)$ is termed the \emph{signature} of the space. 
A given signature, $(p,q)$, implies that we may construct an 
orthogonal basis for the space,
$\{e_i\}$, $i=1,\cdots,n=p+q$ where $e_i^2=+1$ for $i=1,\cdots,p$
and $e_i^2=-1$ for $i=p+1,\cdots,n$; i.e.\ we take a general mixed
signature space.  

To perform geometric operations using GA we now extend the space 
to ${\cala}(p+1,q+1)$ via the inclusion of two additional orthogonal
basis  vectors, $e$ and
$\bar{e}$, such that
%
\[  e^2=+1,\;\;\;\; \bar{e}^2= -1
\]
%

Note that if $x \in {\cala}(p,q)$, then $e\cdot x =
\bar{e}\cdot x = 0$ since $e_i\cdot e=e_i \cdot \bar{e} = 0$
for $i=1,\cdots,n$. We now compose vectors $n$ and
$\bar{n}$ as
%
\[ n = e + \bar{e}  \qquad  \bar{n} = e - \bar{e} 
\label{nequations}
\]
%
%\begin{eqnarray}
% n & = & e + \bar{e}  \qquad  \bar{n} = e - \bar{e} \\
% \mbox{and thus }\; e & = & \frac{1}{2} (n + \bar{n})  \qquad   \bar{e}  =  \frac{1}{2} (n - \bar{n})
% \label{nequations}
% \end{eqnarray}
% %
These vectors will be useful later. It is easy to see that
$n$ and $\bar{n}$ are \emph{null} vectors since
%
\begin{eqnarray}
n^2  & = & (e + \bar{e})\cdot(e + \bar{e}) = e^2 + 2e\cdot\bar{e} + \bar{e}^2 \nn \\
       & = &  1 + 0 - 1 = 0  \nn
\end{eqnarray}
and
\begin{eqnarray}
\bar{n}^2  & = & (e - \bar{e})\cdot(e - \bar{e}) = e^2 - 2e\cdot\bar{e} + \bar{e}^2 \nn \\
       & = &  1 - 0 - 1 = 0     \nn
\end{eqnarray}
%
We also note a number of useful identities for $n$, $\bar{n}$ and $x \in \cala(p,q)$
%
\begin{eqnarray}
n\cdot\bar{n} & = & (e + \bar{e})\cdot(e - \bar{e}) = e^2 - \bar{e}^2 = 2 \nn \\
x\cdot n = x\cdot \bar{n} & = & 0 \nn 
\end{eqnarray}
%
It is equally easy to show that if we define the bivector $E = n \wedge \bar{n}$ 
then $E^2 = 4$
%
\begin{eqnarray}
  E^2 & = &  (n\wedge \bar{n})\cdot(n\wedge \bar{n}) \nn \\
  & = &  (n\cdot\bar{n})(n\cdot\bar{n}) - n^2 \bar{n}^2 \nn \\
  & = &  4
  \end{eqnarray}
  %
since $n\cdot\bar{n}=2$ and $n^2=\bar{n}^2=0$.

To perform geometric operations upon vectors, points and objects
in $\cala(p,q)$ we map a point $x \in \cala(p,q)$ to a point
$F(x) \in \cala(p+1,q+1)$ and find that the operations may be performed
by simple algebraic manipulation of $F(x)$. The specific mapping
used is the Hestenes (\cite{HS84}, p.302) representation
%
\begin{equation}
F(x)=-\frac{1}{2}(x-e)n(x-e)
\end{equation}
%
where, substituting for $n=e+\bar{e}$ and using the fact that
$\bar{e}\cdot x = 0 = \bar{e}\cdot e = n\cdot x$, it is not
hard to rewrite this equation in terms of the null
vectors as follows
%
\begin{equation}
  F(x) = \frac{1}{2}(x^2n + 2x - \bar{n})
\end{equation}
%
which is similar to the form which is used in the more recent
`horosphere' formulations of the conformal framework
\cite{oldwine}. We will see that there is some choice as
to what factor we put in front of the $x^2n + 2x -
\bar{n}$ expression; we choose $\half$ so that our
normalisation condition for null vectors, which allows us to
compute the reverse mapping $F(x) \mapsto x$ independent of the
absolute scale of $F(x)$, becomes
%
\[ F(x)\cdot n = -1  \]
%

We will see that it will often be necessary to work with
normalised `unit' lines, planes, circles and spheres in order to
apply the reflection etc. formul\ae\ which will be so
important in later sections. It is also worth noting that the
mapping as it stands above is dimensionally inconsistent. In later
chapters it will be shown how correcting this can allow
one to extend the approach to non-Euclidean geometries.

It is somewhat trivial to show that $F(x)$ is always a null vector for
any $x$ by evaluating $[F(x)]^2$
%
\begin{eqnarray}
 [F(x)]^2 & = &  \frac{1}{4}(x^2n + 2x - \bar{n})\cdot (x^2n + 2x - \bar{n}) \nn \\
              & = & -\frac{1}{2}x^2 n\cdot \bar{n} + x^2  \nn \\
              & = &  -x^2 + x^2 = 0
\end{eqnarray}
%

Thus we have mapped vectors in ${\cala}(p,q)$ into \emph{null}
vectors in ${\cala}(p+1,q+1)$ and this is precisely the horosphere 
construction. It also shows the need to impose a normalisation
constraint upon the resultant null vectors as they remain null 
irrespective of absolute scale.
More generally we can show that all null vectors in $\cala(p+1,q+1)$
must be the result of mapping some vector $x \in \cala(p,q)$ as above;
any vector $X \in {\cala}(p+1,q+1)$ can be written as
%
\[ X = an + bx + c\bar{n}  \]
%
where $x = x^ie_i$, $i=1,\cdots,p+q$ and thus $x \in \cala(p,q)$.
We can say that $X\cdot n = 2c$ and $X\cdot \bar{n}=2a$ (since $n$ is 
null and $x\cdot n = 0$). Therefore $a$
and $c$ are uniquely determined. However we can also write our
general $X$ as  $X=b{x}^ie_i + \alpha e + \bar{\alpha}\bar{e}$ for 
suitable scalars $\alpha$ and $\bar{\alpha}$ and
have $X\cdot e_j= bx^j$ ($j=1,\cdots,p+q$). So, whilst the product
$bx^j$ is uniquely defined by $X$, $b$ and $x^j$ individually are
not. Now suppose that $X$ is null so that $X^2=0$
%
\begin{eqnarray}
X^2 & = &   (an + bx + c\bar{n})\cdot (an + bx + c\bar{n}) \nn \\
       & = &  2ac n\cdot \bar{n} + b^2x^2 \nn \\
       & = & b^2x^2 + 4ac = 0
       \label{condition}
\end{eqnarray}
%

From this we are easily able to see that any null vector can be
written in the form
%
\begin{equation}
\lambda (x^2n + 2x - \bar{n}) \label{null}
\end{equation} 
%
since if $(an + bx + c\bar{n}) = \lambda(x^2n + 2x -
\bar{n})$ we have that
%
\[c=-\lambda\;\;\;\; \lambda x^2 = a \;\;\;\mbox{and}\;\;\; 2\lambda = b\]
%
and we can then eliminate $\lambda$ from these last two
equations to give the condition $b^2 x^2 = -4ac$. This
is precisely the condition given in
equation~\ref{condition}.

These results may now be used to provide a projective mapping
between ${\cala}(p,q)$ and  ${\cala}(p+1,q+1)$. Specifically that the
family of null vectors $\lambda(x^2n + 2x - \bar{n})$,
in ${\cala}(p+1,q+1)$ are taken to correspond to the
single point $x\in {\cala}(p,q)$. If $x$ is the origin
then we see that $F(x) = -\bar{n}$ and we therefore may associate
null vectors parallel to $\bar{n}$ with the origin. 
We will see later that when we \emph{invert} $\bar{n}$ 
we obtain $n$, thus suggesting
that we associate null vectors parallel to $n$ with the 
\emph{point at infinity} (the usual result of inverting the origin).

When we look at the inner product of null vectors in
${\cala}(p+1,q+1)$ we discover something very
interesting. If $A$ and $B$ in ${\cala}(p+1,q+1)$
represent the points $a$ and $b$ in ${\cala}(p,q)$, then
%
\begin{eqnarray} A\cdot B &  =  &  F(a)\cdot F(b) \nn \\
            &  =  &  \frac{1}{4}(a^2n + 2a - \bar{n})\cdot (b^2n + 2b -
            \bar{n})\nn \\
            & = &    -\frac{1}{2}a^2 + a\cdot b - \frac{1}{2}b^2 \nn \\
            &  =  &  -\frac{1}{2}(a-b)^2
            \label{distance}
\end{eqnarray}
%
and we see that $A\cdot B$ is proportional to the Euclidean
distance between points $a$ and $b$. This property of the
conformal space and its relationship to \emph{distance geometry}
\cite{distgeom} is discussed at more length in \cite{oldwine}. 
The mapping and properties described here were outlined
originally in \cite{HS84}.

\subsubsection{Rotations}


In normal 3d GA rotations are performed with elements of the algebra termed
\emph{rotors}. In this section we aim to show that these rotors may be used
unchanged on the null vectors we obtain by performing our projective mapping.
Let $x \mapsto Rx\tilde{R}$ with $x\in {\cala}(p,q)$ and $R$ be a
rotor in the Clifford algebra over ${\cala}(p,q)$. Consider
what happens when $R$ acts upon $F(x)$; i.e.\ the nature of
$RF(x)\tilde{R}$
%
\[ RF(x)\tilde{R} =\frac{1}{2} R(x^2n + 2x - \bar{n})\tilde{R} =\frac{1}{2}[ x^2Rn\tilde{R}
+ 2Rx\tilde{R} - R\bar{n}\tilde{R}] \]
%
and since $R$ is a rotor it contains only even blades and therefore 
commutes with
$n$ and $\bar{n}$  
%($e_in=-ne_i$, so if we have an even number of $e_i$s we
%		have commutation), so that $Rn\tilde{R} = R\tilde{R} n = n$ and
%$R\bar{n}\tilde{R} = \bar{n}$. Thus we have
so
%
\begin{equation} RF(x)\tilde{R} = \frac{1}{2}(\hat{x}^2n + 2\hat{x} - \bar{n})
	\end{equation}
%
where $\hat{x}=R x \tilde{R}$. We have therefore shown that rotors in
${\cala}(p,q)$ remain rotors in ${\cala}(p,+1q+1)$ in that they retain
their action about the point $x$ represented by $F(x)$. To summarise
%
\begin{equation} x \mapsto Rx\tilde{R} \qquad \Leftrightarrow \qquad F(x) \mapsto
F(Rx\tilde{R}) \end{equation}
%

\subsubsection{Translators}

Translation along a vector $a$ is defined as the
mapping $x\mapsto x+a$ for some $x,a \in \cala(p,q)$. In this
section we will show that this is performed by applying a 
rotor $R=T_a = \exp{\frac{na}{2}}$ to $F(x)$.
To begin with, consider the usual power series expansion of the exponential
which we may immediately simplify thus
%
\begin{equation}
R=T_a = \exp{\frac{na}{2}} = 1 + \frac{na}{2} +
\frac{1}{2}\left(\frac{na}{2}\right)^2 + \cdots   = 1 +
\frac{na}{2}
\end{equation}
%
since $n$ is null, $an = -na$ and therefore the higher order terms are all
zero. We now see how $R$ acts on the vectors $n$, $\bar{n}$ and $x$.
%
\begin{eqnarray}
 Rn\tilde{R} & = & \left(1 + \frac{na}{2}\right)n\left(1 + \frac{an}{2}\right) \nn \\
                & = &  n + \frac{1}{2}nan + \frac{1}{2}nan + \frac{1}{4}nanan \nn \\
                & = & n
                \label{rntrans}
\end{eqnarray}
%
again using  $an= - na$ and $n^2=0$.    Similarly we can
show that
%
\begin{eqnarray}
 R\bar{n}\tilde{R} & = &  \bar{n}  - 2a - a^2n \\
 Rx\tilde{R}          & = &  x + n(a\cdot x)
\end{eqnarray}
%
Immediately we see that our interpretation of $n$ being the point at
infinity and $\bar{n}$ being the origin is consistent with our claim that
$R$ represents the translation $x \mapsto x+a$ since $R(-\bar{n})\tilde{R} = F(a)$
and the point at infinity is unchanged by finite translation.

We can now also see how the rotor acts on $F(x)$
%
\begin{eqnarray}
RF(x)\tilde{R} & = & \left(1 + \frac{na}{2}\right)\half(x^2n + 2x - \bar{n})\left(1 + \frac{an}{2}\right) \nn \\
                & = &  \half(x^2n + 2(x + n(a\cdot x)) - (\bar{n} - 2a - a^2n))   \nn \\
                & = &  \half((x+a)^2n + 2(x+a) - \bar{n}) \nn \\
                & = &  \half(\hat{x}^2 n + 2\hat{x} - \bar{n}) = F(x+a)
\end{eqnarray}
%
where $\hat{x}=x+a$ and thus translations in
${\cala}(p,q)$ can be performed by the
rotor  $R=T_a$ defined above. To summarise
 %
 \begin{equation}
x \mapsto x+a \qquad \Leftrightarrow \qquad F(x) \mapsto T_a
F(x) \tilde{T}_a = F(x+a)
\end{equation}

\subsubsection{Inversion}


In the usual 3d geometric algebra we can reflect a vector $a$ in a plane with
unit normal $n$ by `sandwiching' the vector between the normal, $-nan$
\cite{IJCV98}. Sandwiching the object \emph{to be reflected} between the object
\emph{in which we wish to reflect} is a very general prescription in GA and
one which will be used heavily in later parts of the report.  In this section
we look at how inversions are brought about by this same reflection operation.

In this report by `inversion' we mean the mapping $x \mapsto \frac{x}{x^2}$ or,
equivalently, $x \mapsto x^{-1}$. Firstly, we look at the properties of
the reflection in $e$ of various vectors
%
\[ -ene  = -ee\bar{n} = -\bar{n} \] 
since $ne = (e+\bar{e})e=(e^2+\bar{e}e)= (e^2 - e\bar{e}) = e\bar{n}$. 
Similarly, we can show that a number of reflection properties hold
%
\begin{eqnarray} -ene & = &  -\bar{n}  \\ -e\bar{n}e & = &  -{n}  \\ -exe & = &
x \end{eqnarray}
%
and finally we may observe what happens to $F(x)$ under reflection in $e$
%
\begin{eqnarray} -eF(x)e & = &  -e\half(x^2n + 2x - \bar{n})e  \nn \\ & = &
\half\left[-x^2\bar{n} + 2x + n\right] \nn \\ & = &  x^2\half\left[
\frac{1}{x^2}n + 2\frac{x}{x^2} - \bar{n}\right] \nn \\ & = &  x^2
F\left(\frac{x}{x^2}\right) \end{eqnarray}
%

We have, therefore, shown that the inversion operation in ${\cala}(p,q)$ can be
performed via the reflection in $e$ of the projection into ${\cala}(p,+1q+1)$.
 %
 \begin{equation} x \mapsto \frac{x}{x^2} \qquad \Leftrightarrow \qquad F(x) \mapsto
 -\frac{eF(x)e}{x^2} = F\left(\frac{x}{x^2}\right) \end{equation}
%

Since the absolute scale of $F(x)$ is irrelevant, as we always rescale to
impose our normalisation constraint, we can omit the scaling by $x^{-2}$.  It
is also irrelevant, by the same logic, whether we take $-e(\cdot)e$ or
$e(\cdot)e$ as the reflection and henceforth we will use $e(\cdot)e$ for
convenience. We reiterate here that later in the report the concept of
reflection of a quantity in an object being brought about by this method
of `sandwiching'
%the
%quantity between the object 
is crucial to much of our code for fast 3d
manipulations.

\subsubsection{Dilators}


A dilation by a factor of $\alpha$ is the mapping
$x \mapsto \alpha x$. In this section we investigate how
to form a rotor which has the action of dilating about the
origin. We start by considering the rotor $R = D_{\alpha} =
\exp{\frac{\alpha}{2}e\bar{e}}$ and a number of
relations which can easily be verified
%
\begin{eqnarray}
   -e\bar{e}n  = &  {n} & = ne\bar{e}  \nn \\
   -\bar{n}e\bar{e} = &  \bar{n} & =  e\bar{e}\bar{n}  \label{eneqn}
\end{eqnarray}
%
We can now look at what $RF(x)\tilde{R}$ gives
%
\begin{eqnarray}
D_\alpha F(x) \tilde{D}_\alpha & = & \exp{\frac{\alpha}{2}e\bar{e}}\half\{ x^2n + 2x - \bar{n}\}\exp{-\frac{\alpha}{2}e\bar{e}} \nn \\
   &  =  &  \half(x^2  \exp{\alpha e\bar{e}}n + 2x - \exp{\alpha e\bar{e}}\bar{n}) \nn \\
   & = &  \half(x^2 \exp{-\alpha} n + 2x - \exp{\alpha}\bar{n}) \nn \\
   & = & \exp{\alpha}\half\left\{\exp{-2\alpha}x^2n + 2\exp{-\alpha}x - \bar{n}\right\} \nn \\
   & = & \exp{\alpha}\half\left\{\hat{x}^2n + 2\hat{x} -\bar{n}\right\}
\label{Deqn}
\end{eqnarray}
%
where $\hat{x} = \exp{-\alpha}x$. The above steps can be
verified by considering $\exp{-\frac{\alpha}{2}e\bar{e}}$ as
the expansion $1 - \frac{\alpha}{2}e\bar{e} +
\frac{1}{2!}\left(\frac{\alpha}{2}e\bar{e}\right)^2 +
\cdots$ and using the relations given in
equation~\ref{eneqn}.
Again noting that the absolute scale of $F(x)$ doesn't matter we
have therefore shown that dilations  in ${\cala}(p,q)$ can be performed by
the rotor    $R=D_\alpha$
 %
 \begin{equation}
x \mapsto \exp{-\alpha}x \qquad \Leftrightarrow \qquad F(x)
\mapsto  D_{\alpha} F(x) \tilde{D}_{\alpha} =
\exp{\alpha}F(\exp{-\alpha}x)
\end{equation}
%
Note that the signs are incorrect in the equivalent
equations in \cite{HS84}, p.303, equation~3.22. It is worth noting
that dilation about any other point may be achieved by concatenating the
appropriate rotors to move that point to the origin, dilate and move back.


\subsubsection{Special conformal transforms}


We have seen above that we are able to express rotations,
inversions, translations and dilations in ${\cala}(p,q)$
by rotations and reflections in  ${\cala}(p+1,q+1)$. This
now leads us to consider \emph{special conformal
transformations}. These are essentially transformations
which preserve angles and are defined by the motion
%
\begin{equation}
x \mapsto  x \frac{1}{1 + ax}
\end{equation}
%

A moment's investigation reveals this transform to be 
a combination of inversion, translation and inversion again
%
\begin{eqnarray}
x   & \stackrel{\longmapsto}{_{\mbox{inversion}}} & \frac{x}{x^2} \nn \\
     &  \stackrel{\longmapsto}{_{\mbox{translation}}} & \frac{x}{x^2} + a  \equiv \frac{x}{x^2}(1+xa)  \nn \\
     &  \stackrel{\longmapsto}{_{\mbox{inversion}}} & \frac{\frac{x}{x^2} + a}{(\frac{x}{x^2} + a)(\frac{x}{x^2} + a)} \nn \\
   & = & \frac{x+ ax^2}{1+2a\cdot x + a^2x^2} = x\frac{1}{1 + ax}
\end{eqnarray}
%
since $\frac{1}{1+ax} = \frac{1+xa}{(1+ax)(1+xa)}$. The
final line in the above expression shows us that
$x\frac{1}{1 + ax}$ is indeed a vector since $x+ ax^2$ is
a vector. As we have built up the special conformal
transformation via inversions and translations, we know
exactly how to construct the ${\cala}(p+1,q+1)$ operator
that performs such a transformation by simply chaining the
rotors for inversion and translation we derived above.
The required rotor is therefore given by
%
\begin{equation}
 K_a = eT_a e,\qquad {\mbox{so that}} \qquad x\mapsto K_a x \tilde{K}_a
\end{equation}
%
and
\begin{equation}
   K_a x \tilde{K}_a = e\left\{ T_a(exe)\tilde{T}_a\right\}e
\end{equation}
%
%Now recall that $T_a = 1+ \frac{na}{2}$, $a\in
%{\cala}(p,q)$, so that $eT_ae = e\{1+\frac{na}{2}\}e =
%e^2 + \frac{1}{2}enae = 1-\frac{1}{2}\bar{n}a$. Thus, we
Substituting for the rotors above we can write our special conformal rotor as
%
\begin{equation}
K_a = 1 - \frac{1}{2}\bar{n}a
\end{equation}
%
We are now in a position to see what happens when we act
on $F(x)$ with $K_a$
%
\begin{eqnarray}
 K_a F(x) \tilde{K}_a & = & eT_a(eF(x)e)\tilde{T}_a e \nn \\
               & = & eT_a( - x^2 F(\frac{x}{x^2}))\tilde{T}_a e \nn \\
       & = &  -x^2 e\left\{ F(\frac{x}{x^2} +a)\right\} e \nn \\
& = & -x^2\left\{- \left(\frac{x}{x^2} + a\right)^2
F\left(\frac{\left(
\frac{x}{x^2} + a\right)}{\left(\frac{x}{x^2} + a\right)^2}\right)\right\} \nn \\
          & = & (1 + 2a\cdot x + a^2x^2)F\left( x\frac{1}{1+ax}\right)
\end{eqnarray}
%
The end result is therefore
%
\begin{equation}
x\mapsto x\frac{1}{1+ax}  \qquad \Leftrightarrow \qquad  F(x)
\mapsto (1+2a\cdot x + a^2x^2)F\left(x\frac{1}{1+ax}\right)
\end{equation}
%

%
\subsection{Observations}
We can see that, from the above, the following results are true:
%
\begin{eqnarray}
R n \tilde{R} &  =  &   n \;\;\; \mbox{for $R$ a
rotation, since} \;\;\; n\tilde{R}=\tilde{R}n
 \nn \\
 R n \tilde{R} &  =  &   n \;\;\; \mbox{for $R$ a
translation (equation~\ref{rntrans}}) \nn \\
n\tilde{R}=\tilde{R}n \;\;\mbox{and}\;\;\; R n \tilde{R}
& = & \exp{(-\alpha )}n \;\;\;\mbox{for dilations} \nn
\end{eqnarray}
%

Thus rotations, translations and dilations leave $n$, which we identify with
the point at infinity, unchanged up to a scale factor. This is a fact which
will be important to us in subsequent sections. Indeed, we find that the
underlying geometry described by the rotors is related to the element of the
algebra which the rotors hold invariant. We will see later that the 5d
conformal setup provides a framework in which we can simply describe
non-Euclidean geometries in such terms.


\subsubsection{Intersections}


It is often found that many problems in computer graphics,
robotics, inverse kinematics and many other fields require
finding the intersections of lines, planes, circles and spheres.
In this chapter we discuss how to use the techniques and theorems
developed in the preceding chapters to find intersections
and thereby show that GA provides a highly elegant framework
for such problems.

Before looking at particular examples we will look
briefly at representations of linear combinations of
point representations. When we represent projective geometry via 4d
homogeneous coordinates we are still able to consider
linear combinations of points if we are willing to make
the appropriate normalisation operations; we would also
like to be able to consider linear combinations of points
in the ${\cala}(p+1,q+1)$ space. We know that a linear
combination of two points $a,b$, in Euclidean space, of
the form
%
\[  \lambda a + \mu b \;\;\;\; \mbox{where} \;\;\;\;\; \lambda + \mu =1  \]
%
gives another point on the line joining $a$ and $b$.
Similarly, a linear combination of 3 points, $a,b,c$ of
the form
%
\[  \lambda a + \mu b  + \nu c  \;\;\;\; \mbox{where} \;\;\;\;\; \lambda + \mu  + \nu=1  \]
%
gives another point on the plane containing $a,b$ and
$c$. The usual projective representation, where we go up
just one dimension, has the advantage of still being
linear in the representations of the points e.g. if
$x=\lambda a + \mu b + \nu c$  (with $\lambda + \mu + \nu
=1$) then its 4d projective representation, $X=x+e$,
where $e$ is the extra basis vector which takes 3d to 4d,
can also be written in the form
%
\[ X = \lambda A + \mu B + \nu C  \]
%
since
%
\begin{eqnarray}
 X  & =  & x+e=\lambda a + \mu b + \nu c +
 (\lambda+\mu+\nu)e \nn \\
 &    = &  \lambda A + \mu B + \nu C  \nn
 \end{eqnarray}
 %
Here we note that we have insisted that the point representation
$X$ is `normalised', i.e. that $X=x + e$ rather than some
multiple of this. With the conformal representation,
working in ${\cala}(p+1,q+1)$, we appear to have lost
this advantage of linearity. For example, if $A$ and $B$
are the ${\cala}(p+1,q+1)$ representatives of $a$ and
$b$, then in general
%
\[ \lambda A + (1-\lambda)B \ne \;\;\mbox{a multiple of}\;\;\; F(\lambda a + (1-\lambda)b)  \]
%
This is due to the presence of the $x^2n$ term in the
representation, which removes linearity. However, the
following \emph{is} true and is easy to show from the
definition of $F(x)$:
%
\begin{equation}
F(\lambda a + (1-\lambda)b) = \lambda A + (1-\lambda)B +
\lambda(1-\lambda)(A\cdot B) n \label{lincomb}
\end{equation}
%
We therefore see that the departure from linear behaviour
is given by the addition of a multiple of the representation of the point at
infinity. This is relatively benign behaviour and means
that many of the techniques we use in the GA version of
projective geometry will still work here. For example,
this gives us another way of seeing that the equation for
a line passing through points $a$ and $b$ is $X\wedge n
\wedge A\wedge B=0$ -- the wedging with $n$ knocks out the
non-linear term $\lambda(1-\lambda)A\cdot B n $ and we are
left with the usual GA projective geometry result.

Precisely the same sort of thing goes through for a
plane. Let $a,b,c$, in 3d,  define a plane and let
%
\[ x = \alpha a + \beta b + \gamma c \;\;\; \mbox{where}\;\;\; \alpha + \beta + \gamma = 1  \]
%
be a general point on the plane. Then it is easy to show
that the representative of $x$, $X=F(x)$ satisfies
%
\begin{equation}
   X = \alpha A + \beta B + \gamma C + \delta n \;\;\mbox{where}\;\;\; \delta = (\alpha\beta A\cdot B + \alpha \gamma A\cdot C + \beta\gamma B\cdot C)
\end{equation}
%
again making it clear why the equation of the plane
can be written as
%
\[ X\wedge  n \wedge A \wedge B \wedge C =0 \]
%
Note that in this chapter, and subsequent chapters  where
we use the same multiples ($\alpha, \beta, \gamma$ etc)
in ${\cala}(p+1,q+1)$ space as in ${\cala}(p,q)$ space,
it is important that the representatives are taken as
$F(a)$, $F(b)$ etc and not arbitrary multiples of these.
What it amounts to is that we have to enforce
our usual normalisation constraint
%
\begin{equation}
 X\cdot n  = -1
 \end{equation}
 %
 
We have already seen that when looking at the dual representation
of lines, planes, circles and spheres, it is useful to work with
these normalised representations. So far we have avoided using the
term homogeneous (a variety of other approaches to the conformal
model use the term \emph{homogeneous} e.g. \cite{oldwine}):
although we are associating all points $\lambda X$ with a given 3d
point $x$, by insisting upon a normalisation we increase the range
of procedures we can employ. If one were truly to use homogeneous
coordinates then for many operations there would be a variety of
factors to introduce, which would mimic our normalisation. One
argument against using normalised coordinates is that we lose the
usefulness of being able to avoid carrying around large numbers
when dealing with points far from the origin. In practise, we have
not met with any difficulties, but everything in this report can be
reproduced adopting a homogeneous viewpoint and simply introducing
various factors if certain operations are desired. The ease with
which the use of normalised points enables one to think about the
addition of 5d points and its meaning persuades us that we should
maintain this viewpoint for the theoretical analysis we are
presenting. In practise, the points need not be stored as
normalised points.

 This formalism for dealing with linear combinations
of 5d points has some interesting properties. Given two
null vectors $A$ and $B$ representing 3d points $a$ and
$b$, we know from the above that taking $X =\half (A +
B)$ will give us  $C + \alpha n$ where $C$ is the 5d
representative of the midpoint of the line joining $a$
and $b$, i.e. $c=\half (a+b)$. Extracting $C$ from $X$ is
now a special case of the more general problem of
extracting $P$, where $P^2=0$, from an expression of the
form
%
\[ X = \alpha P + \beta n   \]
%
To do this we first note that $XnX={\alpha}^2PnP$ since
sandwiching $n$ gets rid of the $\beta n$ term. Secondly,
we note that in taking $PnP$ we are effectively \emph{reflecting} 
$n$ in the point representation $P$. It is to be expected
that this can produce only one thing, which is a multiple
of the point itself. Indeed this is the case, and it is
not difficult to prove that
%
\[  PnP = -2P  \]
%
Thus using the fact that $X\cdot n = \alpha P\cdot n =
-\alpha$, we see that $P = -(XnX)/2\alpha^2$ leading to
%
\begin{equation}
  P =  \frac{-(XnX)}{2(X\cdot n)^2}
\end{equation}
%
We will use the above extraction procedure a number of
times in subsequent constructions.

We now go on to look at some explicit examples of
intersecting objects in this framework. Firstly we look
at the general \emph{meet} operation, familiar from
projective geometry, and how it appears in geometric
algebra. Suppose we are intersecting two objects $W_r$
and $W_s$ (an $r$-blade and an $s$-blade representing $r$
and $s$-grade objects), if $X$ lies on the intersection
of $W_r$ and $W_s$ then we know that
%
\[ X\wedge W_r = 0 \;\;\;\mbox{and}\;\;\;\; X\wedge W_s = 0
\]
%
It can then be shown that
%
\begin{equation}
X\cdot \left< W_r W_s \right>_{2n-r-s} = 0 \;\;\; \mbox{or equivalently }
X\wedge \{\left[\left< W_r W_s \right>_{2n-r-s}\right]I_n\} = 0
\label{newmeet}
\end{equation}
%
where $n$ is the dimension of the space, in this case, $n=5$. Thus
the intersection of the two objects $W_r$ and $W_s$ is given by
taking the $2n-r-s$ grade part of the product of $W_r$ and $W_s$.
%(the proof of this is given elsewhere FIXME no it isn't)
This is equivalent to the form of the meet given in
\cite{HZ91,HS84},
%
\begin{equation}
W_r \vee W_s = \{ W_r^* \wedge W_s^*\}^*
\end{equation}
%
where the operator $\vee$ denotes the meet or
intersection operation and the $^*$ refers to taking the
dual with respect to the \emph{join} of the two objects
(e.g. in the case of the intersection of 2 lines, the
join is not the whole space and we would not therefore be
able to use our pseudoscalar $I_5$ for evaluating the
dual in the above expression for the meet). In the case
where the join is indeed the whole space and we can use
the pseudoscalar $I_5$, we can also write the meet as
%
\[W_r \vee W_s = W_r \cdot (W_s)^*  \]
%
Clearly this need for evaluating the join is cumbersome, and the
formula given in equation~\ref{newmeet} is far preferable as it
can be computed with ease. We will now go on to see how useful
this formalism is in intersecting objects -- it is a generic
formula regardless of the nature of our objects (from within the
set of spheres, circles, planes and lines), and can thus be
programmed up very easily.



\section{Intersecting spheres with spheres or planes}

Let us consider the intersection of two spheres, $\Sigma_1$ and
$\Sigma_2$ where they may intersect in a circle, at a point or not
at all. Suppose we take the formula for the meet (where now we use
$*$ to indicate multiplication by the pseudoscalar $I_n$):
%
\begin{equation}
C = \Sigma_1 \vee \Sigma_2 = \left[\left< \Sigma_1 \Sigma_2
\right>_{2n-r-s}\right]^*
\end{equation}
%
$2n-r-s = 10-4-4=2$, so that the dual quantity will have
grade $5-2=3$ -- generally, this will give the trivector
representing the circle of intersection. We can tell
whether we have a circle, a point intersection or no
intersection according to whether
%
\begin{equation}
C^2 >0 \;\;\;\; C^2 = 0 \;\;\;\mbox{or}\;\;\; C^2<0
\end{equation}
%
In the case of $C^2>0$ we can extract the centre and radius
according to equation~\ref{centre_radius}. If we also attempt to
extract the centre and radius via these same formul\ae\ from $C$
where $C^2=0$, we will find that the circle will have zero radius
and its centre will be the point of tangency of the two spheres.
Similarly, attempting to extract the radius and centre from $C$ in
the case $C^2<0$ (i.e. no intersection) leads to an imaginary
radius and a centre which lies on the shortest line joining the
surfaces of the spheres (i.e. that joining the centres). If the
two spheres have the same radii, it is the midway point on this
line.

The above all follows through if instead of having a
second sphere, $\Sigma_2$, we have a plane, $\Phi_2$ --
we again get a trivector for our intersection object via the meet and the sign
of the square of this trivector tells us whether the two objects are tangent,
intersect in a circle or do not intersect at all.

Although we will not go into this here, it is interesting
to note that in the case of the non-intersecting objects,
we can also extract information such as the perpendicular
vector between the surfaces from the product $W_rW_s$.


\subsection{The method of projectors}
\label{sec:projectors}

Some intersections discussed below, namely line-sphere and line-circle give, in general, two points of
intersection. In these cases we find that the meet generates a new class of object, a
bivector representing a pair of points; the points $A$ and $B$ are simply represented
by $A \wedge B$. It is simple to show that the null-space of this has the 
required form since
\[
(A \wedge B) \wedge X = A \wedge (B \wedge X) =  0
\]
is only true in general if $B = X$ and, by symmetry, if $A = X$. 

Extracting $A$ and $B$ from $A \wedge B$ requires a little care and may be preformed 
using the method of projectors.

Let our bivector be $T = A \wedge B$. Firstly consider the effect of taking the
inner product of $T$ with $n$ 
%\textbf{FIXME}: Should I explain this more
%fully?
\[
T \cdot n = -n \cdot T = -(n\cdot A)B + (n\cdot B)A = B-A
\]
which is true as the identity $n \cdot F(x) = -1$ can be shown by 
direct substitution. Now form
\[
F = \frac{1}{\beta}A \wedge B
\]
where $\beta = T^2$ and hence $F^2=1$ (we shall consider the special 
case $\beta =0$ later). We now define two \emph{projector} operators in terms of $F$
\[
P = \frac{1}{2}(1 + F)
\]
\[
\tilde{P} = \frac{1}{2}(1 - F)
\]

As suggested by the operator naming, $\tilde{P}$ is the reverse of $P$ (the reverse of a 
bivector $F$ is simply $-F$). These operators also have the following interesting 
property
\begin{eqnarray*}
P^2 & = & \frac{1}{4} (1+F)(1+F) \\
    & = & \frac{1}{4} (1 + 2F + F^2) \\
    & = & \frac{1}{4} (2 + 2F) \\
    & = & \frac{1}{2} (1 + F) = P
\end{eqnarray*}
and similarly $\tilde{P}^2 = \tilde{P}$. It is also trivial to show that $P\tilde{P}=0$ and
that
\[
PA = \tilde{P}B = 0
\]
\[
PB = B, \  \tilde{P}A = A
\]
We can now extract $A$ and $B$ from $T$
\begin{equation}
P \left[ T\cdot n \right] = P (B - A) = PB-PA = B\label{eqn:extractB}
\end{equation}
\begin{equation}
\tilde{P} \left[ T \cdot n \right] = \tilde{P} (B-A) = \tilde{P}B - \tilde{P}A = -A
\label{eqn:extractA}
\end{equation}

Remember that this approach was only valid if $\beta^2 \ne 0$. Let's consider the
form of $\beta_2$ a little more 
\begin{eqnarray*}
\beta & = & (A \wedge B)(A \wedge B) \\
      & = & \frac{1}{4}(AB - BA)(AB - BA) \\
      & = & \frac{1}{4}(ABAB - ABBA - BAAB + BABA) \\
      & = & 0 \quad\mbox{ iff } ABAB + BABA = ABBA - BAAB
\end{eqnarray*}
The condition for $\beta = 0$ can be simplified further
\begin{eqnarray*}
ABAB + BABA & = & ABBA - BAAB \\
            & = & AB^2A - BA^2B \\
	    & = & 0 - 0 = 0 \\
\Rightarrow ABAB & = & -BABA 
\end{eqnarray*}

This condition is only satisfied in general if $A = B$ and hence 
if there is only one point of intersection, i.e.\ the line
is tangential to the sphere. A similar type of analysis can be used to
show that there is no intersection if $\beta^2 < 0$.

We thus have a general method for factorising $A \wedge B$.

\section{Intersecting spheres with circles or lines}

Let us now intersect a sphere $\Sigma_1$ (4-blade) with a
circle $C_2$ (3-blade). According to our meet formul\ae
our intersection is a 2-blade, $B$, given by
%
\begin{equation}
B = \Sigma_1 \vee C_2 = \left[\left< \Sigma_1 C_2
\right>_{2n-r-s}\right]^*
\end{equation}
%
where $2n-r-s=10-4-3=3$, so that the dual object has
grade 2. We have already seen that these 2-blades
represent 2 points -- precisely as we would expect, since
an intersecting sphere and circle will do so at 2 points.
Now, again, we look at the sign of the resulting 2-blade,
$B$, and we will find that there are two, zero or one
point of intersection according to
%
\[  B^2>0 \;\;\;\; B^2=0 \;\;\;\mbox{or}\;\;\;\; B^2<0
\]
%
We have seen earlier that given a bivector $B$, such that
$B^2>0$, of the above form, we can extract the two points
of intersection via the projectors given in
equations~\ref{eqn:extractB} and \ref{eqn:extractA}. Now, if $B^2=0$ we cannot form
the projector, but it is trivial to find the representation of the point of
intersection, $X$, in this case using the following
%
\[  X = BnB  \]
%
i.e. for a 2-blade of the form $W = P\wedge Q$, reflecting
$n$ in $W$, $WnW$, would give us the midpoint of the line
joining $P$ and $Q$ -- for our case where $B^2=0$, the
construction $BnB$ will therefore give us a representation of the point of
intersection. These results can easily be shown by
considering simple cases at the origin and then extending
the proof via rotors as previously.

Precisely the same working holds if we replace our
circles above with lines -- the meet again gives a
2-vector whose square tells us whether there are 2, 1 or
no intersections, and from which the intersection points
can be obtained easily. We will return to the
intersections of lines with spheres when we later
consider reflections of lines in spheres.

Again, in the non-intersecting case the product $W_rW_s$
can provide us with information about the vectors between
the surfaces.




\section{Intersecting planes with planes, circles and lines }

Consider two planes $\Phi_1$ and $\Phi_2$; taking the
meet gives
%
\begin{equation}
L = \Phi_1 \vee \Phi_2 = \left[\left< \Phi_1 \Phi_2
\right>_{2n-r-s}\right]^*
\end{equation}
%
where $2n-r-s=10-4-4=2$, so that the dual object has
grade 3 -- as we would expect, if the planes intersect to
give a line. We are able to tell whether the planes
intersect by looking at the sign of $L^2$ -- if $L^2=0$
we know that the planes are parallel and do not
intersect, if $L^2>0$, the planes intersect in the line
$L$.

Now consider a plane $\Phi_1$ and a circle $C_2$; we take
the meet of these two objects to give
%
\begin{equation}
B = \Phi_1 \vee C_2 = \left[\left< \Phi_1 C_2
\right>_{2n-r-s}\right]^*
\end{equation}
%
where $2n-r-s=10-4-3=3$, so that the dual object has
grade 2 -- the plane and the circle intersect in a
maximum of two points, and the 2-blade, $B$ encodes these
two points as with the sphere-circle intersection. Once
again we can assert that there are 2,1 or 0 intersections
according to whether $B^2>0,B^2=0,B^2<0$. In the case of
two intersections, the points are extracted from $B$ by
projectors as before, and in the case of tangency, the
one point of contact is obtained by taking $BnB$.

It is worth thinking about what happens when the circle
$C_2$ lies in the plane $\Phi_1$ so that the intersection is
$C_2$ itself. As one might expect, in this case there \emph{is
no} grade 3 part of $\Phi_1C_2$. In the case of the $z=0$ plane and
the unit circle lying in the plane and centred on the origin this can
easily be confirmed:

\[
\Phi_1 = F(e_1) \wedge F(e_2) \wedge F(-e1) \wedge n \propto e_1e_2e\bar{e}
\]
\[
C_2 = F(e_1) \wedge F(e_2) \wedge F(-e1) \propto e_1e_2\bar{e}
\]
thus
\[
\Phi_1C_2 \propto e_1e_2e\bar{e}e_1e_2\bar{e} = e \mbox{ hence }
\left<\Phi_1C_2\right>_3 = 0
\]

We can also note that, in this case, the dual of $\Phi_1C_2$ with respect to
$e_1e_2e\bar{e}$ is indeed $C_2$.

If we now replace our circle by a line, $L_2$, it is
clear that the meet will still give us a 2-blade, $B$;
but we know that the line and the plane intersect in at
most one position, so should we not be looking for a
vector rather than a 2-blade? The answer is that if the
plane and the line intersect, and the meet gives us $B$,
then $B$ is always of the form
%
\[  B = X\wedge n  \]
%
where $X$ is the representation of the point of intersection. This can be
proved easily by again considering a simple case at the
origin. If $B^2>0$ the line and plane intersect in a
point, if $B^2=0$ the line and plane do not intersect and
if $B=0$ the line lies in the plane. If there is one
point of intersection so that $B$ is of the above form,
we can extract the 3d point of intersection,
$x=x^ie_i,\;i=1,2,3$ (and hence $X$), by simply equating
$x^i$ to the coefficient of the $e_i\wedge n$ term or by
using the following expansion
%
\begin{equation}
 x = (B\wedge \bar{n})\cdot E
 \end{equation}
%
where $E=n\wedge \bar{n}$ as given earlier.


\section{Intersecting circles with circles and lines }

Consider two circles, $C_1$ and $C_2$, taking their meet
gives
%
\begin{equation}
X = C_1 \vee C_2 = \left[\left< C_1 C_2
\right>_{2n-r-s}\right]^*
\end{equation}
%
where $2n-r-s=10-3-3=4$, so that the dual object has
grade 1. We know, however, that the intersection of two
circles has at most 2 intersections (only possible if
they lie in the same plane), so how do we get two
intersections from our grade 1 object? In fact we find
that the following is true
%
\begin{eqnarray}
C_1 \vee C_2  &  =  &  X \;\;\mbox{where}\;\;
X^2=0\;\;\mbox{if circles have one intersection} \nn \\
C_1 \vee C_2  &  =  &  X \;\;\mbox{where}\;\;
X^2\ne 0\;\;\mbox{if circles have no intersection} \nn \\
C_1 \vee C_2  &  =  &  0 \;\; \;\;\mbox{if circles have
two intersections}
\end{eqnarray}
%
In the case where the meet gives zero and we know there
are two intersections, these can easily be found by
intersecting the plane of one of the circles with the
other circle, i.e.
%
\begin{equation}
B = C_1 \vee (C_2\wedge n) =  \left[\left< C_1 (C_2\wedge n)
\right>_{2n-r-s}\right]^*
\end{equation}
%
where $2n-r-s=10-3-4=3$, so that the dual object has
grade 2, and the two points of intersection can be
extracted from the 2-blade $B$ using
equations~\ref{eqn:extractB} and \ref{eqn:extractA}.

If we now replace $C_2$ by a line $L_2$ we see that we
again get a grade 1 object when we take the meet, and the
situation above is exactly replicated, i.e.
%
\begin{eqnarray}
C_1 \vee L_2  &  =  &  X \;\;\mbox{where}\;\;
X^2=0\;\;\mbox{if circle and line have one intersection} \nn \\
C_1 \vee L_2  &  =  &  X \;\;\mbox{where}\;\;
X^2\ne 0\;\;\mbox{if circle and line have no intersection} \nn \\
C_1 \vee L_2  &  =  &  0 \;\; \;\;\mbox{if circle and
line have two intersections}
\end{eqnarray}
%
As before, in the case where the meet gives zero the two
intersections can easily be found by intersecting the
plane of one of the circles with the line. It is also
interesting to note here that in the case where the
circle and the line do not intersect, with the meet
giving a vector, $X$, which is not null, the sign of
$X^2$ tells us whether the line passes through the circle
($X^2<0$) or does not pass through the circle ($X^2>0$)
-- such simple checks can often be useful in graphics
applications.


Given that we have had a little difficulty with circles
intersecting circles, we might expect some slight
difficulties with lines and lines. It turns out that many
interesting constructions emerge when we start to
consider the intersections between two lines, these will
be discussed in the following section.


\section{Intersecting lines with lines }

Let us consider two lines, $L_1$ and $L_2$. Taking the
meet of these two lines gives
%
\begin{equation}
X = L_1 \vee L_2 = \left[\left< L_1 L_2
\right>_{2n-r-s}\right]^*
\end{equation}
%
where $2n-r-s=10-3-3=4$, so that the dual object has
grade 1. We might expect that if the lines intersect at a
point, the meet, $X$, will give this intersection point
-- however, this is not the case. We find that the
following is true
%
\begin{eqnarray}
L_1 \vee L_2 & = & 0 \;\;\; \mbox{if the lines intersect}
\nn \\
L_1 \vee L_2 & \propto & n \;\;\; \mbox{if the lines do
not intersect}
\end{eqnarray}
%
We therefore have a simple of way for checking for
intersecting lines, but if the meet gives us zero so that
we know there is an intersection point, we can no longer
find this point in a fully covariant way by intersecting
one line with the a plane defined by the other line,
since such a plane is not uniquely defined. We could, in
practise, intersect one line with the plane formed by the
other line and the origin, but then if the other line
passes through the origin, this will not work, and we are
left with a non-covariant procedure and one which entails
us forming conditionals for a number of cases. We would
instead like to look for a method which works co-variantly
-- such a method exists, and in the process of describing
it, we see a number of other useful constructions.

Again take our arbitrary lines, $L_1$ and $L_2$ (assume
they are normalised, such that $L_1^2=L_2^2=1$). Suppose
we reflect line $L_1$ in line $L_2$ -- this statement is
not well-defined in a conventional sense, but in
GA, we have seen that reflection of an
object in another object is indeed well defined and is
brought about by sandwiching the object to be reflected
between the object that it is being reflected in. So, our
reflected line, $L_1'$ is given by
%
\[  L_1' = L_2L_1L_2  \]
%
The operation of reflection is grade preserving if we are
dealing with blades, and therefore we know that we get
another line. This construction will work for any two
lines, but let us now suppose that our lines intersect at
a point, we can then expect that the reflected line
$L_1'$ will be the line formed by the intersection point,
represented by $P$, and any point on $L_1$ reflected in $L_2$. This is
indeed what $L_1'$ is. Now, however, it becomes possible
very easily to form the line which is perpendicular to
$L_2$, passing through the intersection point, $P$, and
in the plane defined by $L_1$ and $L_2$ via the following
%
\begin{equation}
 L_1'' =  L_1 - L_2L_1L_2
\end{equation}
%
%-- this is illustrated in figure~\ref{intersect_lines}
%
%\begin{figure}
%\centerline{
%\includegraphics[width=.4\textwidth]{c:/jl/siggraph/2002/intersect_lines}
%} \caption[]{The rotation of intersecting lines to
%produce two lines intersecting at right angles, via the
%construction $L_1-L_2L_1L_2$.} \label{intersect_lines}
%\end{figure}
%
\begin{figure}
\centerline{
\includegraphics[width=.4\textwidth]{lines}
} \caption{The rotation of intersecting lines to
produce two lines intersecting at right angles, via the
construction $L_1-L_2L_1L_2$.} \label{intersect_lines}
\end{figure}
Clearly what we are doing here is rotating one line in
the plane defined by the two lines,  to be perpendicular
to the other line. We will return to this rotor
description later.  Now we have two perpendicular lines
which intersect, we can find the point of intersection
relatively easily. Take any arbitrary point representation $X$ and
reflect it in $L_1''$ (assuming again that we have
normalised $L_1''$) via $X' = L_1''XL_1''$, then take the
midpoint of $X$ and $X'$, we know that this must lie on
the line $L1'$
%
\[ X' = L_1''XL_1'' \;\;\;\; X'' = \half (X + X')   \]
%
(recall that $X''$ will be our real midpoint plus some
multiple of $n$). Now we reflect $X''$ in $L_2$ to give $X'''$ and again
take the midpoint -- the midpoint must now give us the
intersection point $P$ plus some multiple of $n$;
%
\[ X''' = L_2X''L_2 \;\;\;\;  P' = \half (X''+X''')   \]
%
we then extract the null vector corresponding to the representation of our
intersection point $P$ via
%
\[ P = \frac{-(P'nP')}{2(P'\cdot n)^2}   \]
%
This construction is independent of $X$ and is a
beautiful illustration of the ability to manipulate
objects using geometric algebra. Although it appears
involved, it can be programmed up very easily ($\bar{n}$
can be used as the point $X$) and is an entirely
covariant way of intersecting two lines. Of course, in
practise, one can also check to see if there is at least
one line that does not pass through the origin
($\bar{n}\wedge L=0$ if $L$ passes through the origin), if
there is, say this is $L_1$, we can then form the plane
$\bar{n}\wedge L_1$ and intersect this with $L_2$, if both
lines pass through the origin then the intersection point
is the origin. Note that we can use precisely the same
type of argument to extract the plane formed by two
intersecting lines. For example, take any arbitrary point
$X$, reflect it in the line $L_1$ via $L_1 X L_1$, so
that the midpoint of this line is given by $P$ where $P$
is given (up to some additional multiple of $n$) by
%
\[  P = \half (X + L_1 X L_1)   \]
%
$P$ must clearly lie on $L_1$, thus the plane formed by
the two lines must be given by
%
\begin{equation}
 L_2\wedge P = L_2 \wedge \half(X + L_1X L_1)
\end{equation}
%
Again, $\bar{n}$ can be used for our point $X$ in real
computations. The derivations given here are entirely
covariant and the same constructions will intersect
`lines' in different geometries, we will illustrate this
towards the end of the report by considering 3d hyperbolic
geometry.

 Recall that
previously we found the reflection of $L_1$ in $L_2$ via
$L_2L_1L_2$, now note that we can rewrite this as
%
\begin{equation}
 L_2L_1L_2 = (L_2L_1)L_1\widetilde{(L_2L_1)} = RL_1\tilde{R}
\end{equation}
%
since $\widetilde{(L_2L_1)} = \tilde{L}_1\tilde{L}_2=L_1L_2$.
Thus we see that the quantity $L_2L_1$ acts as a rotor
(if the lines are normalised) which rotates through twice
the angle between the lines about an axis through the
intersection point and perpendicular to the plane
containing the lines. In fact, when we write this
reflection as a rotation, there is nothing to insist that
our lines must intersect. If $L_1$ and $L_2$ do not
intersect, then $L_2L_1L_2$ will still perform a {\em
reflection of $L_1$ in $L_2$}, but we are now able to
interpret exactly what this reflection means for
non-intersecting lines if we regard the operator as a
rotor. It turns out that the rotor $L_2L_1$ is the
product of a rotation rotor and a translation rotor --
the rotation is in the plane normal to the common
perpendicular of the lines and the translation is along
the common perpendicular so that one line is taken onto
and then through the other. We can show that it is
possible to write the product $L_2L_1$ as
%
\begin{equation}
L_2L_1 = (\cos{\theta} + \hat{B}\sin{\theta})(1 + dn)
\end{equation}
%
where $d$ is the 3d vector representing the length and
direction of the common perpendicular, $\hat{B} =
\hat{d}I_3$ (with $\hat{d}=d/\sqrt(d^2)$) and  $\theta$
is the angle between the lines as measured when
translated to lie in the same plane along $d$. We see
that the above is a combination of a rotation rotor and a
translation rotor, we will see later that rotors which
take one object (of the same grade) into another object
are often formed in the way we have outlined here for
lines.

\section{Existing implementations}

One of the goals of this PhD was to continue to develop a software
library designed to help with the implementation of CGA-based algorithms
in an efficient manner. Before work started on designing 
the software, several existing systems 
were investigated. All of the following packages were designed to
provide high-level access to numerical computations using GA.

\subsection{CLUCalc \& CLUDraw}


CLU \& CLUDraw were written by Christian Perwass and may be obtained from his
web-site (http://www.perwass.de/cbup/clu.html). Of all the systems, this
is the only one designed both for CGA and the visualisation of spheres,
circles, etc.\ directly from the CGA model.

It is written in C++ and uses the object-oriented features of the language
extensively. Multivectors are represented as objects and operations upon
them are performed by overloading the standard operators of the C++
language. 

CLU is a library designed for numerical computations and is not limited
to the signature used for the conformal model but also has support for
other signatures. CLUDraw is a library designed to take multivectors calculated
by CLU and provide a convenient way to visualise them as spheres, lines, planes,
etc. 

Although it provides a convenient interface, the heavy use of C++ object-orientation
and operator overloading within the library results in a rather high
computational overhead. The decoupling of the calculation engine and
visualisation engine however provides the useful ability to remove the
graphics code in a clean manner.


\subsection{GABLE}

TBC

\subsection{Gaigen}

The Gaigen 
homepage\footnote{http://carol.wins.uva.nl/$\tilde{\ }$fontijne/gaigen/} describes 
it as
\begin{quote}
Gaigen is a program which can generate implementations of geometric algebras. It generates C++, C and assembly source code which implements a geometric algebra requested by the user. People who are new to geometric algebra may think that there is only one geometric algebra. However, there are many different geometric algebras. The properties that make these algebras different are, among others, their dimensionality and the signature of their basis vectors. Each of these different algebras may be useful for different applications.
\end{quote}
The user can select the signature of the space and generate C-code to implement it. The
basic code is quite na\"ive (similar to the initial code created for
this project, see later) insofar as it
performs the full $O(1024)$ matrix multiplication required. It does allow the
user to specify a number of special purpose optimised routines to
find the geometric product of, for example, a bivector and trivector.
Gaigen doesn't possess a visualisation engine by default.

\begin{table}
\centering
\textbf{Geometric Product Multiplication Table}\\
\rule{0cm}{0.3cm}
%Geometric Product Multiplication Matrix:\\
%\begin{equation}\left[\begin{array}{cccccccc}
%+A_{0} & +A_{1} & +A_{2} & +A_{3} & -A_{12} & -A_{13} & -A_{23} & -A_{123}\\
%+A_{1} & +A_{0} & +A_{12} & +A_{13} & -A_{2} & -A_{3} & -A_{123} & -A_{23}\\
%+A_{2} & -A_{12} & +A_{0} & +A_{23} & +A_{1} & +A_{123} & -A_{3} & +A_{13}\\
%+A_{3} & -A_{13} & -A_{23} & +A_{0} & -A_{123} & +A_{1} & +A_{2} & -A_{12}\\
%+A_{12} & -A_{2} & +A_{1} & +A_{123} & +A_{0} & +A_{23} & -A_{13} & +A_{3}\\
%+A_{13} & -A_{3} & -A_{123} & +A_{1} & -A_{23} & +A_{0} & +A_{12} & -A_{2}\\
%+A_{23} & +A_{123} & -A_{3} & +A_{2} & +A_{13} & -A_{12} & +A_{0} & +A_{1}\\
%+A_{123} & +A_{23} & -A_{13} & +A_{12} & +A_{3} & -A_{2} & +A_{1} & +A_{0}
%\end{array}\right]
%\end{equation}

%Geometric Product Multiplication Table:\\
\begin{tabular}{|r||c|c|c|c|c|c|c|c|}
\hline
 &
1 & $e_{1}$ & $e_{2}$ & $e_{3}$ & $e_{12}$ & $e_{13}$ & $e_{23}$ & $e_{123}$\\
\hline
\hline
1 & +1 & +$e_{1}$ & +$e_{2}$ & +$e_{3}$ & +$e_{12}$ & +$e_{13}$ & +$e_{23}$ & +$e_{123}$\\
\hline
$e_{1}$ & +$e_{1}$ & +1 & +$e_{12}$ & +$e_{13}$ & +$e_{2}$ & +$e_{3}$ & +$e_{123}$ & +$e_{23}$\\
\hline
$e_{2}$ & +$e_{2}$ & -$e_{12}$ & +1 & +$e_{23}$ & -$e_{1}$ & -$e_{123}$ & +$e_{3}$ & -$e_{13}$\\
\hline
$e_{3}$ & +$e_{3}$ & -$e_{13}$ & -$e_{23}$ & +1 & +$e_{123}$ & -$e_{1}$ & -$e_{2}$ & +$e_{12}$\\
\hline
$e_{12}$ & +$e_{12}$ & -$e_{2}$ & +$e_{1}$ & +$e_{123}$ & -1 & -$e_{23}$ & +$e_{13}$ & -$e_{3}$\\
\hline
$e_{13}$ & +$e_{13}$ & -$e_{3}$ & -$e_{123}$ & +$e_{1}$ & +$e_{23}$ & -1 & -$e_{12}$ & +$e_{2}$\\
\hline
$e_{23}$ & +$e_{23}$ & +$e_{123}$ & -$e_{3}$ & +$e_{2}$ & -$e_{13}$ & +$e_{12}$ & -1 & -$e_{1}$\\
\hline
$e_{123}$ & +$e_{123}$ & +$e_{23}$ & -$e_{13}$ & +$e_{12}$ & -$e_{3}$ & +$e_{2}$ & -$e_{1}$ & -1
\\
\hline
\end{tabular}


\caption{Example \TeX\ output from Gaigen\label{tab:gaigen_output}}
\end{table}

Gaigen does have the useful ability to generate product tables for the algebra
(see Table \ref{tab:gaigen_output}) in both plain text and \TeX\ format. 
%Indeed
%some of these tables were used as input to the code-generating 
%PERL\footnote{Pathologically Eclectic Rubbish Lister---see http://www.perl.org/} 
%scripts.


\subsection{Cambridge GA library for Maple}

This library\cite{GA:CambridgeGALibrary} provides Geometric Algebra
capabilities for the Maple V and VI symbolic mathematics package. It provides
no visualisation capabilities above those provided by Maple. This is a very
useful tool for research but is aimed more at symbolic manipulation than
numerical computation.

\section{Existing uses}

TBC

\cite{DBLP:conf/giae/WarehamCL04}

\subsection{Motion Capture}
\subsection{Computer Vision}
\subsection{Physics}
