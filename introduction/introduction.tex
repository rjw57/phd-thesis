\begin{savequote}
\quoteperson{You know we all became mathematicians for the same reason: 
  we were lazy.}{Max Rosenlicht}
\end{savequote}

\chapter{Introduction}
\label{chap:introduction}

It is generally accepted that a four-dimensional projective description of
three-dimensional Euclidean geometry can have various advantages, particularly
when intersections of planes and lines are required. Such projective
descriptions are used extensively in computer vision and graphics where
rotations and translations are usually described by a single $4\times 4$
matrix. Since its inception in the mid-1970s, computer graphics (CG) has
almost universally used linear vector algebra as its mathematical framework.
This is due primarily to two factors; most early practitioners of computer
graphics were mathematicians familiar with it and linear algebra provided a
compact, efficient way of representing points, transformations, lines, etc.

In the early-1980s CG moved out of the realm of Computer Science research and
started to be used in the broader scientific community as an important
research tool both for simulation and visualisation.  Computer Graphics was
then, and to an extent still is, tied to classical vector algebra which has
started to show a number of problems when applied to the problems being
investigated. Amongst these problems were poor generalisation to spaces other
than $\mathbb{R}^3$, great conceptual difficulty in extending problems to
non-Euclidean geometries and manipulating geometric objects other than simple
lines, planes and points.

As computing power becomes cheaper, the opportunity arises to investigate new
frameworks for CG which, although perhaps not providing the time and space
efficiency of vector algebra, may provide a conceptually simpler system or one
of greater analytical power. This thesis investigates the suitability of
Geometric Algebra (GA) as one such approach. As one might hope the original
four-dimensional description of projective geometry fits very nicely into the
Geometric Algebra framework\cite{Hx91,IJPR00}. 

\section{A Brief Overview of Geometric Algebra}

As stated above, classical vector algebra has a number of problems once one
moves away from three-dimensional Euclidean space. Perhaps the clearest
example is the cross-product of two vectors; the product, $\mathbf{a} \times
\mathbf{b}$, is conventionally defined as a vector normal to the plane
containing $\mathbf{a}$ and $\mathbf{b}$ and has magnitude $ab\sin\theta$
where $\theta$ is the angle between $\mathbf{a}$ and $\mathbf{b}$. However the
normal to the plane is only uniquely defined in three-dimensions and has no
meaning in 2- or 1-dimensional space; the cross-product does not generalise to
higher-dimension spaces. The product is an important element of vector algebra
and one can see that performing geometric operations in higher-spaces without
it quickly becomes complex.

\subsection{The products}

It was in an attempt\cite{GA:grassmann} to create an algebra of vectors which
did generalise to higher spaces that a German schoolteacher named Hermann
Gra{\ss}mann (1809--1877) created an \emph{exterior} or \emph{outer} product
of two vectors denoted as $a \wedge b$. For the remainder of this thesis we
have dropped the usual convention of emboldening vectors since in GA they lie
in the same algebra as scalars and do not need to be differentiated; the
nature of the element is either stated explicitly or clear from context.

Gra{\ss}mann's outer product is usually visualised geometrically as the
movement of one vector along the other to form a `directed area'. This is a
new object, neither a vector nor a scalar. It is termed a \emph{bivector}.
Similarly one may form the outer product of this bivector with another vector
to form a directed volume, a \emph{trivector}, or generally a $n$-volume
termed an $n$-vector. 

To differentiate between scalars, vectors, bivectors, etc we say that a scalar
is grade 0, a vector is grade 1, a bivector (formed from two vectors) is grade
2, etc. A $n$-vector has grade $n$.

\begin{figure}
\centering
\includegraphics{geometric}
\caption{Illustration of bi- and trivectors\label{fig:geometric}}
\end{figure}

The usual geometric visualisation is illustrated in figure \ref{fig:geometric}.
It is worth noting that other visualisations may be more appropriate for a specific
application so the reader should not assume a bivector can \emph{only} represent a
directed area. 

A key feature of GA is that the outer-product is anti-commutative and
associative giving
\begin{displaymath}
a \wedge b = - ( b \wedge a)\quad\mbox{and}\quad
a \wedge (b \wedge c) = (a \wedge b) \wedge c = a \wedge b \wedge c.
\end{displaymath}

Most of Gra{\ss}mann's work was largely ignored by the mathematical community.
It was not until William Clifford (1845--1879) investigated Gra{\ss}mann's
algebra in 1878\cite{GA:clifford} that the crucial step which made GA a useful
algebra was made.

Clifford unified Gra{\ss}mann's outer product and the familiar dot
or inner-product into one \emph{geometric product} such that
\begin{displaymath}
ab = a\cdot b + a \wedge b.
\end{displaymath}
An algebra with this product is usually termed a \emph{Clifford algebra}. We
shall use the term Geometric Algebra to mean the \emph{coupling} of Clifford
algebras with an accompanying geometric interpretation.

A second glance at the geometric product shows an interesting feature which
should be noted. In the expression above we are adding a scalar ($a \cdot b$)
to a bivector ($a \wedge b$). That is we are adding a grade 0 element to a
grade 2 element. This combination of differing grade objects is analogous to
complex numbers where we linearly combine a real and imaginary number to form
a complex number. In this case we refer to such a combination of objects of
varying grade as a \emph{multivector}.  As noted earlier we shall refer to all
single-grade elements with lower-case letters but use upper-case letters to
refer to multivectors. The geometric product also gives us a convenient new
definition of the outer and inner products for vectors as
\[
A \wedge B = \frac{1}{2}(AB - BA)\;\mbox{and}
\]
\[
A \cdot B = \frac{1}{2}(AB + BA).
\]

The power of this approach may be illustrated through its application to
rotation. In two dimensions this is easily performed using complex numbers;
representing the vector $[x\ y]$ as the complex number $z = x + iy$, rotation
by $\theta$ radians can be performed by multiplication with $e^{i\theta}$.
William Hamilton (1805--1865) worked for many years to extend this approach to
three-dimensions. He eventually created \emph{quaternions}
\cite{hamilton2,hamilton1}, an algebra with 4 basis elements $\left\{1,
\mathbf{i}, \mathbf{j}, \mathbf{k}\right\}$ from which all elements are
generated through linear combination. This algebra, although functional,
lacked an obvious geometrical interpretation and again didn't generalise
easily to higher-dimensions.

\subsection{Rotation via Rotors}

To demonstrate Clifford's approach, consider any three orthonormal basis
vectors of $\mathbb{R}^3$, $\left\{e_1, e_2, e_3\right\}$. We can form
3 different bivectors from these vectors:
\begin{displaymath}
B_1 = e_2e_3,\quad B_2 = e_3e_1,\quad B_3 = e_1e_2.
\end{displaymath}

Note that these are indeed bivectors since the basis vectors are orthogonal and
\[
e_ie_j = e_i \cdot e_j + e_i \wedge e_j = e_i \wedge e_j \quad \mbox{iff} \quad i \ne j
\]

Now consider the effect of $B_3$ on the vectors $e_1$ and $e_1 + e_2$:
\begin{displaymath}
e_1B_3 = e_1e_1e_2 = e_1^2e_2=e_2 
\end{displaymath}
\begin{displaymath}
(e_1 + e_2)B_3 = e_1B_3 + e_2B_3 = e_2 + e_2e_1e_2 = e_2 - e_1e_2^2 = e_2 - e_1
\end{displaymath}

\begin{figure}
\centering
\includegraphics{rotation}
\caption{The rotation effect of bivector $B_3 = e_1e_2$\label{fig:rotation}}
\end{figure}

By looking at figure \ref{fig:rotation} it is clear that $B_3$ has the effect
of rotating the vectors counter-clockwise by 90 degrees. It is, in fact, a
general property that the bivector $e_ie_j$ will rotate a vector 90 degrees in
the plane defined by $e_i$ and $e_j$. At first glance this seems to offer
little more than quaternions but at no point have we assumed that we are
working in three-dimensional space; this method also works in higher-dimension
spaces.

We can also easily extend to general rotations; it is trivial to
show that $B_3$ squares to $-1$:
\begin{displaymath}
B_3^2 = e_1e_2e_1e_2 = -e_1e_2e_2e_1 = -1
\end{displaymath}

We can represent any vector $x$ in the plane defined by $e_1$ and
$e_2$ using
\begin{eqnarray*}
x & = & r ( e_1 \cos \theta + e_2 \sin \theta) \\
  & = & e_1 r ( \cos \theta + B_3 \sin \theta)
\end{eqnarray*}
where $r$ is the distance of the point $x$ from the origin (i.e.\ $r = \sqrt{x^2}$)
and $\theta$ is the angle $x$ makes to $e_1$. Also note
that $e_1B_3 = e_1^2e_2 = e_2$. By taking the Taylor expansion of cosine
and sine and re-arranging the coefficients it can be shown that
\[
e^{B_3\theta} = \cos \theta + B_3 \sin \theta
\]
which is the GA analogue of de Moivre's theorem for complex
numbers.

We can thus represent any vector $x$ which lies in the plane of the
bivector $B_3$ by
\[
x=e_1re^{B_3\theta}
\]

From this the same argument used for rotation in the complex plane 
can be used to show that rotation by
$\phi$ radians in the plane of $B_3$ is accomplished by $x \mapsto x'$
where
\begin{displaymath}
x' = xe^{B_3\phi} = x (\cos \phi + B_3 \sin \phi)
\end{displaymath}

This has all taken place in two dimensions for the moment but nothing in our
discussion has assumed this. In fact we can specify a bivector $B_3 = ab$ in
three dimensions and rotate vectors in the place defined by $a$ and $b$ using
the expression above.

\begin{figure}
\centering
\includegraphics{rotation2}
\caption{Rotating vectors in arbitrary planes\label{fig:rotation2}}
\end{figure}

Careful consideration must be given to the case where the vector to be
rotated, $x$, does not lie on the plane of rotation as in figure
\ref{fig:rotation2}. Firstly decompose the
vector into a component which lies in the plane $x_\parallel$ and one
normal to the plane $x_\perp$
\[
x = x_\parallel + x_\perp
\]

Now consider the effect of the following
\begin{eqnarray*}
e^{-B_3\phi/2}
x
e^{B_3\phi/2}
& = & \left(\cos \frac{\phi}{2} - B_3 \sin \frac{\phi}{2}\right)
(x_\parallel + x_\perp )
\left(\cos \frac{\phi}{2} + B_3 \sin \frac{\phi}{2}\right) \\
& = & x_\parallel (\cos \phi + B_3 \sin \phi) + x_\perp
\end{eqnarray*}
since bivectors anti-commute with vectors in their plane (e.g. 
$e_1(e_2e_1) = -e_2 = -(e_2e_1)e_1$) and commute with
vectors normal to the plane (e.g.\ $e_1(e_2e_3) = (e_2e_3)e_1$).
We have thus succeeded in rotating the component of the vector 
which lies in the plane without affecting the component normal
to the plane --- we have rotated the vector around an axis normal to
the plane.

This leads to a general method of rotation in any plane; we
form a bivector of the form $R = \exp({-B\phi/2})$ for a given rotation 
$\phi$ in a plane specified by the bivector $B$. The transformation
is therefore performed by
\begin{displaymath}
x \mapsto RxR^{-1}.
\end{displaymath}
We refer to these bivectors which have a rotational effect as \emph{rotors}.
Figure \ref{fig:rotation2} shows the various objects used. Later we shall
extend the term \emph{rotor} to refer to an element of the algebra which
performs some well defined transformation.

Computing $R^{-1}$ is rather difficult analytically and can sometimes
require a full $2^n$-dimension matrix inversion for a space of
dimension $n$. To combat this we define
the \emph{reversion} of a $n$-vector $X = e_ie_j...e_k$ as
\[
\tilde{X} = e_k...e_je_i
\]
i.e. the literal reversion of the components. By looking at the expression for
$R$ it is clear that $\tilde{R} \equiv R^{-1}$ for rotors. Computing
$\tilde{R}$ is easier since it generally just involves changing the sign of
components when an element is resolved onto a set of basis elements.

Note that in spaces with dimension $n$, the maximum grade object possible is
an $n$-grade one. We denote the $n$-vector $e_1 \wedge ... \wedge e_n = I$ as
the \emph{pseudoscalar} and the product $xI$ as the \emph{dual} of a vector or
$x^*$. The dual is similarly defined for general multivectors. The
pseudo-scalar is so termed because it commutes with all elements of the
algebra.

\subsection{Relation to quaternions}
\label{sec:quaternions}

It is worth comparing this method of rotation to rotation via
quaternions. The three bivectors $B_1,B_2$ and $B_3$ act identically to the
three imaginary components of quaternions, $\mathbf{i}, -\mathbf{j}$ and
$\mathbf{k}$ respectively. The sign difference between $B_2$ and $\mathbf{j}$
is due to the fact that the quaternions are not derived from the usual
right-handed orthogonal co-ordinate system. This handedness mismatch often
leads to annoying sign errors in quaternion-based algorithms.

Using quaternions, a particular rotation is represented via the quaternion 
$q$ given by
\[
q = q_0 + q_1 \mathbf{i} + q_2 \mathbf{j} + q_3 \mathbf{k}
\]
where $q_0^2 + q_1^2 + q_2^2 + q_3^2 = 1$. Interpolation between rotations 
represented
by $\{ q_1, \cdots, q_i \}$ is then performed by interpolating between
each quaternion, $q_i$,
over the surface of a four-dimensional hyper-sphere. For example, if we 
were to interpolate
between unit quaternions $q_0$ and $q_1$ the SLERP interpolation would be
\[
q = \left\{
\begin{array}{ll}
q_0(q_0^{-1}q_1)^\lambda & {\rm ~if~} q_0 \cdot q_1 \ge 0 \\
q_0(q_0^{-1}(-q_1))^\lambda & {\rm ~otherwise}
\end{array}
\right.
\]
where $\lambda$ varies in the range $(0,1)$ \cite{slerp}.

Recall that the locus of $\exp(i\theta)$ is the unit
circle. It is straight forward to show that, for some normalised
bivector $B$, the locus of the action of $\exp(B\theta)$ upon a point with respect to varying
$\theta$ is also a circle in the plane of $B$. Hence, if we consider
some rotations $R_1, R_2 = \exp(kB)R_1$, where $k$ is a scalar and
$B$ is some normalised bivector, it is clear that
the quaternionic interpolation is exactly given by
\[
R_\lambda = (R_2 \tilde{R}_1)^\lambda R_1 = \exp(\lambda kB) R_1
\]
where $\lambda$, the interpolation parameter, varies in the range $(0,1)$. A
further moment's thought will reveal that this method is not confined to three
dimensions, like quaternionic interpolation, but instead readily generalises
to higher-dimensions.

\subsection{The Conformal Model}

In the Conformal Model\cite{GA:NFCM} we extend the space by adding two
additional basis vectors. The notation we use will follow the original
notation given in \cite{HS84}. Let $x$ be a vector in a space denoted
${\cala}(p,q)$.  The annotation $(p,q)$ shall be termed the \emph{signature}
of the space.  A given signature, $(p,q)$, implies that we may construct an
orthogonal basis for the space, $\{e_i\}$, $i=1,\cdots,n=p+q$ where $e_i^2=+1$
for $i=1,\cdots,p$ and $e_i^2=-1$ for $i=p+1,\cdots,n$; i.e.\ we take a
general mixed signature space.  

To perform geometric operations using GA we now extend the space 
to ${\cala}(p+1,q+1)$ via the inclusion of two additional orthogonal
basis  vectors, $e$ and
$\bar{e}$, such that
%
\[  e^2=+1,\;\;\;\; \bar{e}^2= -1
\]
%

Note that if $x \in {\cala}(p,q)$, then $e\cdot x =
\bar{e}\cdot x = 0$ since $e_i\cdot e=e_i \cdot \bar{e} = 0$
for $i=1,\cdots,n$. We now compose vectors $n$ and
$\bar{n}$ as
%
\[ n = e + \bar{e}  \qquad  \bar{n} = e - \bar{e}.
\label{nequations}
\]
These vectors will be useful later. It is easy to see that
$n$ and $\bar{n}$ are \emph{null} vectors since
%
\begin{eqnarray}
n^2  & = & (e + \bar{e})\cdot(e + \bar{e}) = e^2 + 2e\cdot\bar{e} + \bar{e}^2 \nn \\
       & = &  1 + 0 - 1 = 0  \nn
\end{eqnarray}
and
\begin{eqnarray}
\bar{n}^2  & = & (e - \bar{e})\cdot(e - \bar{e}) = e^2 - 2e\cdot\bar{e} + \bar{e}^2 \nn \\
       & = &  1 - 0 - 1 = 0.     \nn
\end{eqnarray}
%
We also note a number of useful identities for $n$, $\bar{n}$ and $x \in \cala(p,q)$:
%
\begin{eqnarray}
n\cdot\bar{n} & = & (e + \bar{e})\cdot(e - \bar{e}) = e^2 - \bar{e}^2 = 2 \nn \\
x\cdot n = x\cdot \bar{n} & = & 0. \nn 
\end{eqnarray}
%
It is equally easy to show that if we define the bivector $E = n \wedge \bar{n}$ 
then $E^2 = 4$
%
\begin{eqnarray}
  E^2 & = &  (n\wedge \bar{n})\cdot(n\wedge \bar{n}) \nn \\
  & = &  (n\cdot\bar{n})(n\cdot\bar{n}) - n^2 \bar{n}^2 \nn \\
  & = &  4
  \end{eqnarray}
  %
since $n\cdot\bar{n}=2$ and $n^2=\bar{n}^2=0$.

In the conformal model we use null-vectors to represent points and 
build up objects.
In $\cala(p,q)$ we map a point $x \in \cala(p,q)$ to a vector
$F(x) \in \cala(p+1,q+1)$. Using this representation we find that 
complex geometric operations may be performed
by simple algebraic manipulation of $F(x)$. The specific mapping
used is the Hestenes (\cite{HS84}, page 302) representation
%
\begin{equation}
F(x)=-\frac{1}{2}(x-e)n(x-e)
\end{equation}
%
where, substituting for $n=e+\bar{e}$ and using the fact that
$\bar{e}\cdot x = 0 = \bar{e}\cdot e = n\cdot x$, it is not
hard to rewrite this equation in terms of the null
vectors as follows
%
\begin{equation}
  F(x) = \frac{1}{2}(x^2n + 2x - \bar{n})
\end{equation}
%
which is similar to the form which is used in the more recent
`horosphere' formulations of the conformal framework
\cite{oldwine}. We will see that there is some choice as
to what factor we put in front of the $x^2n + 2x -
\bar{n}$ expression; we choose $\half$ so that our
normalisation condition for null vectors, which allows us to
compute the reverse mapping $F(x) \mapsto x$ independent of the
absolute scale of $F(x)$, becomes
%
\[ F(x)\cdot n = -1.  \]
%

We will see that it will often be necessary to work with
normalised `unit' lines, planes, circles and spheres in order to
apply the various formul\ae\ which will be so
important in later sections. It is also worth noting that the
mapping as it stands above is dimensionally inconsistent. In following
chapters it will be shown how compensating for this inconsistency 
can allow one to extend the approach to various non-Euclidean geometries.

It is possible to show that $F(x)$ is always a null vector for
any $x$ by directly evaluating $[F(x)]^2$
%
\begin{eqnarray}
 [F(x)]^2 & = &  \frac{1}{4}(x^2n + 2x - \bar{n})\cdot (x^2n + 2x - \bar{n}) \nn \\
              & = & -\frac{1}{2}x^2 n\cdot \bar{n} + x^2  \nn \\
              & = &  -x^2 + x^2 = 0
\end{eqnarray}
%

We have mapped vectors in ${\cala}(p,q)$ into \emph{null}
vectors in ${\cala}(p+1,q+1)$ and this is precisely the horosphere 
construction. It also shows the need to impose a normalisation
constraint upon the resultant null vectors as they remain null 
irrespective of absolute scale.
More generally we can show that all null vectors in $\cala(p+1,q+1)$
must be the result of mapping some vector $x \in \cala(p,q)$ as above;
any vector $X \in {\cala}(p+1,q+1)$ can be written as
%
\[ X = an + bx + c\bar{n}  \]
%
where $x = x^ie_i$, $i=1,\cdots,p+q$ and hence $x \in \cala(p,q)$.
We can say that $X\cdot n = 2c$ and $X\cdot \bar{n}=2a$ (since $n$ is 
null and $x\cdot n = 0$). Therefore $a$
and $c$ are uniquely determined. However we can also write our
general $X$ as  $X=b{x}^ie_i + \alpha e + \bar{\alpha}\bar{e}$ for 
suitable scalars $\alpha$ and $\bar{\alpha}$ and
have $X\cdot e_j= bx^j$ ($j=1,\cdots,p+q$). So, whilst the product
$bx^j$ is uniquely defined by $X$, $b$ and $x^j$ individually are
not. Now suppose that $X$ is null so that $X^2=0$
%
\begin{eqnarray}
X^2 & = &   (an + bx + c\bar{n})\cdot (an + bx + c\bar{n}) \nn \\
       & = &  2ac n\cdot \bar{n} + b^2x^2 \nn \\
       & = & b^2x^2 + 4ac = 0
       \label{condition}
\end{eqnarray}
%

From this we are easily able to see that any null vector can be
written in the form
%
\begin{equation}
\lambda (x^2n + 2x - \bar{n}) \label{null}
\end{equation} 
%
since if $(an + bx + c\bar{n}) = \lambda(x^2n + 2x -
\bar{n})$ we have that
%
\[c=-\lambda\;\;\;\; \lambda x^2 = a \;\;\;\mbox{and}\;\;\; 2\lambda = b\]
%
and we can then eliminate $\lambda$ from these last two
equations to give the condition $b^2 x^2 = -4ac$. This
is precisely the condition given in
equation~\ref{condition}.

These results may now be used to provide a projective mapping
between ${\cala}(p,q)$ and  ${\cala}(p+1,q+1)$. Specifically that the
family of null vectors $\lambda(x^2n + 2x - \bar{n})$,
in ${\cala}(p+1,q+1)$ are taken to correspond to the
single point $x\in {\cala}(p,q)$. If $x$ is the origin
then we see that $F(x) = -\bar{n}$ and we may therefore associate
null vectors parallel to $\bar{n}$ with the origin. 
We will see later that when we \emph{invert} $\bar{n}$ 
we obtain $n$, suggesting
that we associate null vectors parallel to $n$ with the 
\emph{point at infinity} (the usual result of inverting the origin).

When we look at the inner product of normalised null vectors in
${\cala}(p+1,q+1)$ we discover something very
interesting. If $A$ and $B$ in ${\cala}(p+1,q+1)$
represent the points $a$ and $b$ in ${\cala}(p,q)$, then
%
\begin{eqnarray} A\cdot B &  =  &  F(a)\cdot F(b) \nn \\
            &  =  &  \frac{1}{4}(a^2n + 2a - \bar{n})\cdot (b^2n + 2b -
            \bar{n})\nn \\
            & = &    -\frac{1}{2}a^2 + a\cdot b - \frac{1}{2}b^2 \nn \\
            &  =  &  -\frac{1}{2}(a-b)^2
\end{eqnarray}
%
and we see that $A\cdot B$ is related to the Euclidean
distance between points $a$ and $b$. 
We can therefore define the Euclidean distance between two
point representations as
\begin{equation}
d(A,B) = \sqrt{-2 (A \cdot B) }
            \label{distance}
\end{equation}
This property of the
conformal space and its relationship to \emph{distance geometry}
\cite{distgeom} is discussed at more length in \cite{oldwine}. 
The mapping and properties described here were outlined
originally in \cite{HS84}.

\subsubsection{Rotations}

In usual descriptions of GA (without the use of the conformal model)
rotations are performed with elements of the algebra termed
\emph{rotors}. In this section we aim to show that these rotors may be used
unchanged on the null vectors point-representations.
Let $x \mapsto Rx\tilde{R}$ with $x\in {\cala}(p,q)$ and $R$ be a
rotor in the Geometric Algebra over ${\cala}(p,q)$. Consider
what happens when $R$ acts upon $F(x)$; i.e.\ the nature of
$RF(x)\tilde{R}$
%
\[ RF(x)\tilde{R} =\frac{1}{2} R(x^2n + 2x - \bar{n})\tilde{R} =\frac{1}{2}[ x^2Rn\tilde{R}
+ 2Rx\tilde{R} - R\bar{n}\tilde{R}] \]
%
and since $R$ is a rotor it contains only even blades and therefore 
commutes with
$n$ and $\bar{n}$  
%($e_in=-ne_i$, so if we have an even number of $e_i$s we
%		have commutation), so that $Rn\tilde{R} = R\tilde{R} n = n$ and
%$R\bar{n}\tilde{R} = \bar{n}$. Thus we have
so
%
\begin{equation} RF(x)\tilde{R} = \frac{1}{2}(\hat{x}^2n + 2\hat{x} - \bar{n})
	\end{equation}
%
where $\hat{x}=R x \tilde{R}$. We have therefore shown that rotors in
${\cala}(p,q)$ remain rotors in ${\cala}(p+1,q+1)$ in that they retain
their action about the point $x$ represented by $F(x)$. To summarise
%
\begin{equation} x \mapsto Rx\tilde{R} \qquad \Leftrightarrow \qquad F(x) \mapsto
F(Rx\tilde{R}) \end{equation}
%

\subsubsection{Translators}

Translation along a vector $a$ is defined for our purposes as the
mapping $x\mapsto x+a$ for some $x,a \in \cala(p,q)$.
In this
section we will show that this is performed by applying a 
rotor $R=T_a = \exp\left({\frac{na}{2}}\right)$ to $F(x)$.

Before proceeding with the proof we should note that in some non-Euclidean
geometries the addition operator, when viewed as a translation operator, is
non-commutative; translation of geodesic $A$ along geodesic $B$ is not the
same, in general, as translation of $B$ along $A$. For the moment we shall
ignore this but it will become important again in later chapters on 
non-Euclidean geometries.

Returning to the form of the translation rotor, consider the usual power series 
expansion of the exponential which we may immediately simplify to
%
\begin{equation}
R=T_a = \exp\left({\frac{na}{2}}\right) = 1 + \frac{na}{2} +
\frac{1}{2}\left(\frac{na}{2}\right)^2 + \cdots   = 1 +
\frac{na}{2}
\end{equation}
%
since $n$ is null, $an = -na$ and therefore the higher order terms are all
zero. We now see how $R$ acts on the vectors $n$, $\bar{n}$ and $x$.
%
\begin{eqnarray}
 Rn\tilde{R} & = & \left(1 + \frac{na}{2}\right)n\left(1 + \frac{an}{2}\right) \nn \\
                & = &  n + \frac{1}{2}nan + \frac{1}{2}nan + \frac{1}{4}nanan \nn \\
                & = & n
                \label{rntrans}
\end{eqnarray}
%
again using  $an= - na$ and $n^2=0$.    Similarly we can
show that
%
\begin{eqnarray}
 R\bar{n}\tilde{R} & = &  \bar{n}  - 2a - a^2n \\
 Rx\tilde{R}          & = &  x + n(a\cdot x)
\end{eqnarray}
%
Immediately we see that our interpretation of $n$ being the point at
infinity and $\bar{n}$ being the origin is consistent with our claim that
$R$ represents the translation $x \mapsto x+a$ since 
$R(-\bar{n})\tilde{R} = F(a)$
and the point at infinity is unchanged by finite translation.

We can now also see how the rotor acts on $F(x)$
%
\begin{eqnarray}
RF(x)\tilde{R} & = & \left(1 + \frac{na}{2}\right)\half(x^2n + 2x - \bar{n})\left(1 + \frac{an}{2}\right) \nn \\
                & = &  \half(x^2n + 2(x + n(a\cdot x)) - (\bar{n} - 2a - a^2n))   \nn \\
                & = &  \half((x+a)^2n + 2(x+a) - \bar{n}) \nn \\
                & = &  \half(\hat{x}^2 n + 2\hat{x} - \bar{n}) = F(x+a)
\end{eqnarray}
%
where $\hat{x}=x+a$ and thus translations in
${\cala}(p,q)$ can be performed by the
rotor  $R=T_a$ defined above. To summarise
 %
 \begin{equation}
x \mapsto x+a \qquad \Leftrightarrow \qquad F(x) \mapsto T_a
F(x) \tilde{T}_a = F(x+a)
\end{equation}

\subsubsection{Inversion}

In the usual three-dimensional geometric algebra we can reflect a vector $a$
in a plane with unit normal $n$ by `sandwiching' the vector between the
normal, $-nan$ \cite{IJCV98}. Sandwiching the object \emph{to be reflected}
between the object \emph{in which we wish to reflect} is a very general
prescription in GA and one which will be used heavily in later parts of the
report.  In this section we look at how inversions are brought about by this
same reflection operation.

In this report by `inversion' we mean the mapping $x \mapsto \frac{x}{x^2}$ or,
equivalently for non-singular areas, $x \mapsto x^{-1}$. Firstly, we look 
at the reflection in $e$ of various vectors
%
\[ -ene  = -ee\bar{n} = -\bar{n} \] 
since $ne = (e+\bar{e})e=(e^2+\bar{e}e)= (e^2 - e\bar{e}) = e\bar{n}$. 
Similarly, we can show that a number of reflection properties hold
%
\begin{eqnarray} -ene & = &  -\bar{n}  \\ -e\bar{n}e & = &  -{n}  \\ -exe & = &
x \end{eqnarray}
%
and finally we may observe what happens to $F(x)$ under reflection in $e$
%
\begin{eqnarray} -eF(x)e & = &  -e\half(x^2n + 2x - \bar{n})e  \nn \\ & = &
\half\left[-x^2\bar{n} + 2x + n\right] \nn \\ & = &  x^2\half\left[
\frac{1}{x^2}n + 2\frac{x}{x^2} - \bar{n}\right] \nn \\ & = &  x^2
F\left(\frac{x}{x^2}\right) \end{eqnarray}
%

We have, therefore, shown that the inversion operation in ${\cala}(p,q)$ can be
performed via the reflection in $e$ of the projection into ${\cala}(p,+1q+1)$.
 %
 \begin{equation} x \mapsto \frac{x}{x^2} \qquad \Leftrightarrow \qquad F(x) \mapsto
 -\frac{eF(x)e}{x^2} = F\left(\frac{x}{x^2}\right) \end{equation}
%

Since the absolute scale of $F(x)$ is irrelevant, as we always rescale to
impose our normalisation constraint, we can omit the scaling by $x^{-2}$.  It
is also irrelevant, by the same logic, whether we take $-e(\cdot)e$ or
$e(\cdot)e$ as the reflection and henceforth we will use $e(\cdot)e$ for
convenience. This `sandwiching' operation will be a common one in the algorithms
we will describe below.
%We reiterate here that later the concept of
%reflection of a quantity in an object being brought about by this method
%of `sandwiching' will be a very common operation in 
%the
%quantity between the object 
%is crucial to much of our code for fast three-dimensions
%manipulations.

\subsubsection{Dilators}


A dilation by a factor of $\alpha$ is the mapping
$x \mapsto \alpha x$. In this section we investigate how
to form a rotor which has the action of dilating about the
origin. We start by considering the rotor $R = D_{\alpha} =
\exp\left({\frac{\alpha}{2}e\bar{e}}\right)$ and a number of
relations which can easily be verified
%
\begin{eqnarray}
   -e\bar{e}n  = &  {n} & = ne\bar{e}  \nn \\
   -\bar{n}e\bar{e} = &  \bar{n} & =  e\bar{e}\bar{n}  \label{eneqn}
\end{eqnarray}
%
We can now look at what $RF(x)\tilde{R}$ gives
%
\begin{eqnarray}
D_\alpha F(x) \tilde{D}_\alpha & = & \exp\left({\frac{\alpha}{2}e\bar{e}}\right)\half\{ x^2n + 2x - \bar{n}\}\exp\left({-\frac{\alpha}{2}e\bar{e}}\right) \nn \\
   &  =  &  \half(x^2  \exp\left({\alpha e\bar{e}}\right)n + 2x - \exp\left({\alpha e\bar{e}}\right)\bar{n}) \nn \\
   & = &  \half(x^2 \exp\left({-\alpha}\right) n + 2x - \exp\left({\alpha}\right)\bar{n}) \nn \\
   & = & \exp\left({\alpha}\right)\half\left\{\exp\left({-2\alpha}\right)x^2n + 2\exp\left({-\alpha}\right)x - \bar{n}\right\} \nn \\
   & = & \exp\left({\alpha}\right)\half\left\{\hat{x}^2n + 2\hat{x} -\bar{n}\right\}
\label{Deqn}
\end{eqnarray}
%
where $\hat{x} = \exp\left({-\alpha}\right)x$. The above steps can be
verified by considering $\exp\left({-\frac{\alpha}{2}e\bar{e}}\right)$ as
the expansion $1 - \frac{\alpha}{2}e\bar{e} +
\frac{1}{2!}\left(\frac{\alpha}{2}e\bar{e}\right)^2 +
\cdots$ and using the relations given in
equation~\ref{eneqn}.
Again noting that the absolute scale of $F(x)$ doesn't matter we
have therefore shown that dilations  in ${\cala}(p,q)$ can be performed by
the rotor    $R=D_\alpha$
 %
 \begin{equation}
x \mapsto \exp{-\alpha}x \qquad \Leftrightarrow \qquad F(x)
\mapsto  D_{\alpha} F(x) \tilde{D}_{\alpha} =
\exp{\alpha}F(\exp{-\alpha}x)
\end{equation}
%
Note that the signs are incorrect in the equivalent
equations in \cite{HS84}, p.303, equation~3.22. It is worth noting
that dilation about any other point may be achieved by concatenating the
appropriate rotors to move that point to the origin, dilate and move back.


\subsubsection{Special conformal transforms}


We have seen above that we are able to express rotations,
inversions, translations and dilations in ${\cala}(p,q)$
by rotations and reflections in  ${\cala}(p+1,q+1)$. This
now leads us to consider \emph{special conformal
transformations}. These are essentially transformations
which preserve angles and are defined by the motion
%
\begin{equation}
x \mapsto  x \frac{1}{1 + ax}
\end{equation}
%

A moment's investigation reveals this transform to be 
a combination of inversion, translation and inversion again
%
\begin{eqnarray}
x   & \stackrel{\longmapsto}{_{\mbox{inversion}}} & \frac{x}{x^2} \nn \\
     &  \stackrel{\longmapsto}{_{\mbox{translation}}} & \frac{x}{x^2} + a  \equiv \frac{x}{x^2}(1+xa)  \nn \\
     &  \stackrel{\longmapsto}{_{\mbox{inversion}}} & \frac{\frac{x}{x^2} + a}{(\frac{x}{x^2} + a)(\frac{x}{x^2} + a)} \nn \\
   & = & \frac{x+ ax^2}{1+2a\cdot x + a^2x^2} = x\frac{1}{1 + ax}
\end{eqnarray}
%
since $\frac{1}{1+ax} = \frac{1+xa}{(1+ax)(1+xa)}$. The
final line in the above expression shows us that
$x\frac{1}{1 + ax}$ is indeed a vector since $x+ ax^2$ is
a vector. As we have built up the special conformal
transformation via inversions and translations, we know
exactly how to construct the ${\cala}(p+1,q+1)$ operator
that performs such a transformation by simply chaining the
rotors for inversion and translation we derived above.
The required rotor is therefore given by
%
\begin{equation}
 K_a = eT_a e,\qquad {\mbox{so that}} \qquad x\mapsto K_a x \tilde{K}_a
\end{equation}
%
and
\begin{equation}
   K_a x \tilde{K}_a = e\left\{ T_a(exe)\tilde{T}_a\right\}e
\end{equation}
%
%Now recall that $T_a = 1+ \frac{na}{2}$, $a\in
%{\cala}(p,q)$, so that $eT_ae = e\{1+\frac{na}{2}\}e =
%e^2 + \frac{1}{2}enae = 1-\frac{1}{2}\bar{n}a$. Thus, we
Substituting for the rotors above we can write our special conformal rotor as
%
\begin{equation}
K_a = 1 - \frac{1}{2}\bar{n}a
\end{equation}
%
We are now in a position to see what happens when we act
on $F(x)$ with $K_a$
%
\begin{eqnarray}
 K_a F(x) \tilde{K}_a & = & eT_a(eF(x)e)\tilde{T}_a e \nn \\
               & = & eT_a( - x^2 F(\frac{x}{x^2}))\tilde{T}_a e \nn \\
       & = &  -x^2 e\left\{ F(\frac{x}{x^2} +a)\right\} e \nn \\
& = & -x^2\left\{- \left(\frac{x}{x^2} + a\right)^2
F\left(\frac{\left(
\frac{x}{x^2} + a\right)}{\left(\frac{x}{x^2} + a\right)^2}\right)\right\} \nn \\
          & = & (1 + 2a\cdot x + a^2x^2)F\left( x\frac{1}{1+ax}\right)
\end{eqnarray}
%
The end result is therefore
%
\begin{equation}
x\mapsto x\frac{1}{1+ax}  \qquad \Leftrightarrow \qquad  F(x)
\mapsto (1+2a\cdot x + a^2x^2)F\left(x\frac{1}{1+ax}\right)
\end{equation}
%

%
\subsection{Observations}
We can see that, from the above, the following results are true:
%
\begin{eqnarray}
R n \tilde{R} &  =  &   n \;\;\; \mbox{for $R$ a
rotation, since} \;\;\; n\tilde{R}=\tilde{R}n
 \nn \\
 R n \tilde{R} &  =  &   n \;\;\; \mbox{for $R$ a
translation (equation~\ref{rntrans}}) \nn \\
n\tilde{R}=\tilde{R}n \;\;\mbox{and}\;\;\; R n \tilde{R}
& = & \exp{(-\alpha )}n \;\;\;\mbox{for dilations} \nn
\end{eqnarray}
%

Thus rotations, translations and dilations leave $n$, which we identify with
the point at infinity, unchanged up to a scale factor. This is a fact which
will be important to us in subsequent chapters. Indeed, we find that the
underlying geometry described by the rotors is related to the element of the
algebra which the rotors hold invariant. We will see later that the five-dimensional
conformal setup provides a framework in which we can simply describe
non-Euclidean geometries in such terms.

\section{Existing implementations}

%One of the goals of this PhD was to continue to develop a software
As part of the research presented here a library designed to help with the
implementation of CGA-based algorithms in an efficient manner was created.
Before work started on designing the software, several existing systems were
investigated. All of the following packages were designed to provide
high-level access to numerical computations using GA.

\subsection{CLUCalc \& CLUDraw}


CLU \& CLUDraw were written by Christian Perwass and may be obtained from his
web-site\cite{CLU}. Of all the systems, this
is the only one designed both for CGA and the visualisation of spheres,
circles, etc.\ directly from the CGA model.

It is written in C++ and uses the object-oriented features of the language
extensively. Multivectors are represented as objects and operations upon
them are performed by overloading the standard operators of the C++
language. 

CLU is a library designed for numerical computations and is not limited
to the signature used for the conformal model but also has support for
other signatures. CLUDraw is a library designed to take multivectors calculated
by CLU and provide a convenient way to visualise them as spheres, lines, planes,
etc. 

Although it provides a convenient interface, the heavy use of C++ object-orientation
and operator overloading within the library results in a rather high
computational overhead. The decoupling of the calculation engine and
visualisation engine however provides the useful ability to remove the
graphics code in a clean manner.


%\subsection{GABLE}
%
%TBC

\subsection{Gaigen}

The Gaigen 
homepage\cite{Gaigen} describes 
it as
\begin{quote}
Gaigen is a program which can generate implementations of geometric algebras. It generates C++, C and assembly source code which implements a geometric algebra requested by the user. People who are new to geometric algebra may think that there is only one geometric algebra. However, there are many different geometric algebras. The properties that make these algebras different are, among others, their dimensionality and the signature of their basis vectors. Each of these different algebras may be useful for different applications.
\end{quote}
The user can select the signature of the space and generate C-code to implement it. The
basic code is quite na\"ive (similar to the initial code created for
this project, see later) insofar as it
performs the full $O(1024)$ matrix multiplication required. It does allow the
user to specify a number of special purpose optimised routines to
find the geometric product of, for example, a bivector and trivector.
Gaigen doesn't possess a visualisation engine by default.

\begin{table}
\centering
\textbf{Geometric Product Multiplication Table}\\
\rule{0cm}{0.3cm}
%Geometric Product Multiplication Matrix:\\
%\begin{equation}\left[\begin{array}{cccccccc}
%+A_{0} & +A_{1} & +A_{2} & +A_{3} & -A_{12} & -A_{13} & -A_{23} & -A_{123}\\
%+A_{1} & +A_{0} & +A_{12} & +A_{13} & -A_{2} & -A_{3} & -A_{123} & -A_{23}\\
%+A_{2} & -A_{12} & +A_{0} & +A_{23} & +A_{1} & +A_{123} & -A_{3} & +A_{13}\\
%+A_{3} & -A_{13} & -A_{23} & +A_{0} & -A_{123} & +A_{1} & +A_{2} & -A_{12}\\
%+A_{12} & -A_{2} & +A_{1} & +A_{123} & +A_{0} & +A_{23} & -A_{13} & +A_{3}\\
%+A_{13} & -A_{3} & -A_{123} & +A_{1} & -A_{23} & +A_{0} & +A_{12} & -A_{2}\\
%+A_{23} & +A_{123} & -A_{3} & +A_{2} & +A_{13} & -A_{12} & +A_{0} & +A_{1}\\
%+A_{123} & +A_{23} & -A_{13} & +A_{12} & +A_{3} & -A_{2} & +A_{1} & +A_{0}
%\end{array}\right]
%\end{equation}

%Geometric Product Multiplication Table:\\
\begin{tabular}{|r||c|c|c|c|c|c|c|c|}
\hline
 &
1 & $e_{1}$ & $e_{2}$ & $e_{3}$ & $e_{12}$ & $e_{13}$ & $e_{23}$ & $e_{123}$\\
\hline
\hline
1 & +1 & +$e_{1}$ & +$e_{2}$ & +$e_{3}$ & +$e_{12}$ & +$e_{13}$ & +$e_{23}$ & +$e_{123}$\\
\hline
$e_{1}$ & +$e_{1}$ & +1 & +$e_{12}$ & +$e_{13}$ & +$e_{2}$ & +$e_{3}$ & +$e_{123}$ & +$e_{23}$\\
\hline
$e_{2}$ & +$e_{2}$ & -$e_{12}$ & +1 & +$e_{23}$ & -$e_{1}$ & -$e_{123}$ & +$e_{3}$ & -$e_{13}$\\
\hline
$e_{3}$ & +$e_{3}$ & -$e_{13}$ & -$e_{23}$ & +1 & +$e_{123}$ & -$e_{1}$ & -$e_{2}$ & +$e_{12}$\\
\hline
$e_{12}$ & +$e_{12}$ & -$e_{2}$ & +$e_{1}$ & +$e_{123}$ & -1 & -$e_{23}$ & +$e_{13}$ & -$e_{3}$\\
\hline
$e_{13}$ & +$e_{13}$ & -$e_{3}$ & -$e_{123}$ & +$e_{1}$ & +$e_{23}$ & -1 & -$e_{12}$ & +$e_{2}$\\
\hline
$e_{23}$ & +$e_{23}$ & +$e_{123}$ & -$e_{3}$ & +$e_{2}$ & -$e_{13}$ & +$e_{12}$ & -1 & -$e_{1}$\\
\hline
$e_{123}$ & +$e_{123}$ & +$e_{23}$ & -$e_{13}$ & +$e_{12}$ & -$e_{3}$ & +$e_{2}$ & -$e_{1}$ & -1
\\
\hline
\end{tabular}


\caption{Example \TeX\ output from Gaigen\label{tab:gaigen_output}}
\end{table}

Gaigen does have the useful ability to generate product tables for the algebra
(see table \ref{tab:gaigen_output}) in both plain text and \TeX\ format. 
%Indeed
%some of these tables were used as input to the code-generating 
%PERL\footnote{Pathologically Eclectic Rubbish Lister---see http://www.perl.org/} 
%scripts.


\subsection{Cambridge GA library for Maple}

This library\cite{GA:CambridgeGALibrary} provides Geometric Algebra
capabilities for the Maple V and VI symbolic mathematics package. It provides
no visualisation capabilities above those provided by Maple. This is a very
useful tool for research but is aimed more at symbolic manipulation than
numerical computation.

\section{Existing uses}

A number of existing applications of GA have been developed in the
field of graphics and vision\cite{DBLP:conf/giae/WarehamCL04,SahanLasenby} and
indeed much of contemporary physics has been recast using GA 
methods\cite{DoranLasenby} providing a conceptually simpler
framework for further research.

This thesis will build on such work and aim to advance, using Geometric
Algebra, a number of fields in Computer Graphics.

% FIXME: Do some boring leg-work filling this section in.
