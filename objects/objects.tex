\begin{savequote}
\quoteperson{A topologist is one who doesn't know the difference between a doughnut and a coffee cup.}%
{John Kelley}%
\end{savequote}
\chapter{Objects in the Conformal Representation}

In this chapter we will look
at the r\^ole played by bivectors in the conformal model and give some useful
alternative representations of lines, planes, circles and spheres.

Throughout this chapter we shall say that a geometric primitive is
represented by some element $M$ of the algebra if, for all
null-vector representations $X$ which satisfy
\[
X \wedge M = 0,
\]
the point represented by $X$ lies on the object and the representaion of all
points on the object satisfy that relation. We term this an \emph{incidence
relation}.

\section{Vectors and 2-blades}

We have seen that we may use null vectors, $X$, in our 5d space to represent 
points in Euclidean 3-space. In particular we identify $n$ and $\bar{n}$ with 
the point at infinity and the origin respectively. In our 5d space there
clearly exist vectors which do not square to zero. The interpretation of such 
vectors will be discussed later.

The term \emph{blade} in GA is used to refer to quantities which can be written as the wedge
product of vectors. For example a $r$-blade can always be written as
$A_1\wedge A_2 \wedge \cdots \wedge A_r$. It is important to distinguish this
from an $r$-vector which may be any linear combination of $r$-blades. 
This is important since, in 5d, not all bivectors can be written as 2-blades;
for example $e_1e_2 + e_3e_4$ cannot be written in the form $a\wedge b$. 

We start by
postulating that $A \wedge B$ represents the pair of points 
with null-vector representations $A$ and $B$. This is clear when 
we consider the incidence 
relation
\[
X \wedge A \wedge B = 0
\]
which is only true in general for $X=A$ or $X=B$. If we accept that
$A\wedge B$ represents two points then we should develop 
an algorithm to extract the individual null-vectors $A$ and $B$.

\subsection{Extracting $A$ and $B$ from $A\wedge B$}
\label{sec:projectors}

In this section we explain how to extract $A$ and $B$ from
$A\wedge B$ using a method of \emph{projectors}. Throughout we shall assume
that $A$ and $B$ have been normalised so that, for example, $A\cdot n = -1$.
We start by considering the 2-blade $T = A\wedge B$ and form
%
\begin{equation}
  F = \frac{1}{\beta} A\wedge B
\end{equation}
%
where $\beta>0$ and ${\beta}^2 = T^2$, so that $F^2 = 1$
if $\beta^2 \ne 0$. We now use $F$ to define two
\emph{projector} operators
%
\begin{eqnarray}
   P & = & \frac{1}{2}(1+F) \nn \\
   \tilde{P} & = & \frac{1}{2}(1-F)
\end{eqnarray}
%
where $\tilde{P}$ denotes the normal reversion operation applied to
$P$. Note that $P^2 = P$, which can be easily be verified
%
\begin{eqnarray}
  PP & = & \frac{1}{4}(1+F)(1+F) \nn \\
     & = & \frac{1}{4}(1+2F + 1) = \frac{1}{2}(1+F)
\end{eqnarray}
%
Similarly, we can easily show that $\tilde{P}\tilde{P} =
\tilde{P}$. These properties justify calling these
operators \emph{projectors}, borrowing the term from physics. 
An equally important property is that $P\tilde{P}=\tilde{P}P = 0$
which, again, is easy to prove
%
\begin{equation}
   P\tilde{P} = \frac{1}{4}(1+F)(1-F)= \frac{1}{4}(1-1)=0
\end{equation}
%
and similarly for $\tilde{P}P$. We may now see what effect
these projectors have on $A$ and $B$
%
\begin{eqnarray}
 PA & = & \frac{1}{2}\left[1+\frac{1}{\beta}A\wedge
 B\right]A \nn \\
   & = & \frac{1}{2} \left[A + \frac{1}{\beta}(A\wedge
 B)A \right] \nn \\
  & = & \frac{1}{2}\left[ A + \frac{1}{\beta}(A\dt
  B)A\right]
  \nn \\
  & = & \frac{1}{2} (A - A) = 0
\end{eqnarray}
%
since $(A\wedge B)A=(A\wedge B)\dt A = -A^2B+(A\dt B)A=(A\dt
B)A$ ($A^2=0$) and  $A\dt B = -\beta$. This is easily
seen from $\beta^2 = (A\wedge B)\dt (A\wedge B) = -A^2B^2 +
(A\dt B)^2$ and the facts that $A^2 = B^2 = 0$ and $A\dt
B$ must be negative as seen in equation~\ref{distance}.

Using similar working we can also show the results of
$P$ and $\tilde{P}$ acting on $A$ and $B$
%
\begin{eqnarray}
 PA & = & 0 \\
 PB & = & B \\
 \tilde{P}A & = & A  \\
 \tilde{P}B & = & 0
\end{eqnarray}
%
The next step is to consider the vector obtained by
dotting $A\wedge B$ with $n$.
%
\begin{equation}
  (A\wedge B)\dt n = -n\dt (A\wedge B) = -(n\dt A)B + (n\dt
  B)A = (B-A)
\end{equation}
%
using the fact that $A$ and $B$ are normalised points
such that $A\dt n=B\dt n = -1$. It therefore follows that
we have
%
\begin{eqnarray}
  P [(A\wedge B)\dt n] & = & P(B - A) = B \\
 - \tilde{P} [(A\wedge B)\dt n] & = & -\tilde{P}(B - A) = A
\end{eqnarray}
%
We note also that since $AP = \tilde{P}A = A$ 
%(see this
%by expanding $A(A\wedge B)$ rather than $(A\wedge B)A$) 
it follows that
$\tilde{P}A P = \tilde{P}\tilde{P}A = \tilde{P}A$.
Similar relations hold for $BP$ etc., so that we have
%
\begin{eqnarray}
 \tilde{P}AP & = & \tilde{P}A \nn \\
 PA\tilde{P} & = & 0 \nn \\
{P}B\tilde{P} & = & {P}B \nn \\
\tilde{P}BP & = & 0 \nn
\end{eqnarray}
%
which means that we can also write the projections as
two-sided operations.
%
Thus from a 2-blade $A\wedge B$ we can extract the two
points, $A$ and $B$ that it represents via
%
\begin{eqnarray}
A & = & -\tilde{P}\left[(A\wedge B)\dt n\right] \equiv
-\tilde{P}\left[(A\wedge B)\dt n\right]P  \nn
\\
B & = & {P}\left[(A\wedge B)\dt n\right] \equiv
{P}\left[(A\wedge B)\dt n\right]\tilde{P} \label{extractAB}
\end{eqnarray}
%
We will see later that when we perform intersection operations
that yield two points the two points in question can then be found 
using the formul\ae\ in equation~\ref{extractAB}. Usefully we do not 
have to solve a quadratic equation as we would do using
conventional approaches.

\section{Trivectors}
We have already seen that there are two classes of object
represented by trivectors. If $P,Q,R$ are null vectors in our 5d
space representing points in 3d space then trivectors of the form
%
\[ C = P\wedge Q\wedge R \]
%
represent circles. Recall that it is specifically a circle 
passing through points represented by $P,Q$ and $R$. Trivectors of the form
%
\[ L = P\wedge Q\wedge n \]
%
represent lines, specifically that line passing through the
points represented by $P$ and $Q$.

In GA it is often found that the operation of taking the dual, that
is multiplication by the pseudoscalar $I$, is usually useful and
often has physical or geometric significance. The dual of an element
is always just another element of the algebra and does not live
in a separate `dual' or `tangent' space. Below we shall consider
the dual operation with respect to the representations of circles 
and lines.

\subsection{Circles as trivectors}

Here we will show that taking the dual of the trivector representing
a circle gives rise to a useful alternative representation which
naturally encodes both the centre and radius.
Let us first of all work in the plane so that our conformal space is
4-dimensional, and is
${\cala}(3,1)$, having basis vectors $e_1,e_2$ and $e,\bar{e}$. We may
easily extend the method to 3d space or higher dimensions as we have done
in previous chapters. 

We start with the unit circle in the plane and take as three points
on it those shown in figure~\ref{figcircle}.
 %
\begin{figure}
\centerline{
\includegraphics[height=0.25\textheight]{circle}
} \caption{Unit circle with three key points marked}
\label{figcircle}
\end{figure}
%
For any unit length vector, $\hat{x}$, we know that
$F(\hat{x})= \half ( n + 2\hat{x} -\bar{n}) = (\hat{x} + \bar{e})$.
In particular we have
%
\[ F(e_1) \wedge F(e_2) \wedge F(-e_1) = 2e_1e_2\bar{e} \]
%
and hence the trivector $C = 2e_1e_2\bar{e}$ represents the unit circle.
In the plane the pseudoscalar, which we shall write as $I_4$,
is given by $I_4 = e_1e_2e\bar{e}$ and so the dual of
$C$, which we write as $C^{*}$, can easily be shown to be
given by
%
\begin{equation}
 C^{*}=CI_4 = 2e = (n+\bar{n})
 \end{equation}
 %
We know that $X\wedge C=0$ is the incidence relation for the circle  and that
$X \wedge C = 0 $ can be rewritten as
%
\[ X \dt (CI_4) =0  \;\;\; \Leftrightarrow \;\;\; X\dt C^{*} = 0  \]
%
Note that $C^{*}$ is the dual of a trivector in a 4d space and is, therefore,
a vector. This suggests a very useful alternative representation
for a circle (or a sphere when generalised to higher dimensions) which we now
discuss.

We know from equation~\ref{distance} that for any two normalised
point representations $A$ and $B$
%
\begin{equation}
A\dt B =  -\half(a-b)^2 \nn
\end{equation}
%
and thus, if $X$ represents a point on a circle and $B$ represents its centre,
we know that we can write
%
\[  X\dt B = -\half(x-b)^2 \equiv -\half\rho^2  \]
%
where $\rho$ is the radius of the circle. For a
normalised point representation $X$ this implies that
%
\[ X\dt (B - \half\rho^2n) =0 \]
%
since $X\dt n = -1$. Comparing this with $X\dt C^*$ we see that
provided we normalise $C^*$ after taking the dual (so that $C^*\dt
n=-1$), then we find
%
\begin{equation}
   C^* = B - \half\rho^2 n
   \label{circle}
   \end{equation}
%
The vector $C^*$, therefore, encodes in a neat fashion
the centre and radius of the circle in the plane.  

Also note that if we take our $(n+\bar{n})$ as the
representation of the dual of the unit circle centred on the
origin and rotate, translate and dilate the expression,
we get precisely the same result as in
equation~\ref{circle}. We can see this by first
taking our unit circle at the origin, $C_o^*=n+\bar{n}$ and
rotating it. We have seen that rotation rotors leave $n$
and $\bar{n}$ invariant, so $C_o^*$ remains unchanged.
Now dilate this with a dilation rotor
$D_{\alpha}=e^{\frac{\alpha}{2}e\bar{e}}$ where
$\rho=e^{-\alpha}$ is the dilation factor. We have seen
previously that
$D_{\alpha}n\tilde{D}_{\alpha}=e^{-\alpha}n$ and
$D_{\alpha}\bar{n}\tilde{D}_{\alpha}=e^{\alpha}\bar{n}$,
so that
%
\begin{equation}
C_o^* \rightarrow  W = e^{-\alpha}n + e^{\alpha}\bar{n}
\end{equation}
%
Now, translate by $a$ using a translation rotor,
$T_{\alpha}=1+\frac{na}{2}$. Again, using previous results, we
know that $n$ is left invariant and $-\half \bar{n}$ is taken to
$F(a)$, giving us
%
\begin{equation}
W \rightarrow  Z = e^{-\alpha}n -2e^{\alpha}F(a)
\end{equation}
%
Normalising this to give us our new $C^*$, where $C^*\dt
n=-1$, means dividing $Z$ above by $-2e^{\alpha}$ (since
$A=F(a)$ satisfies $A\dt n=-1$) leaving us with
%
\begin{equation}
C^* =  -\half e^{-2\alpha}n + A
\end{equation}
%
and we can now see that this is precisely equation~\ref{circle}
with $A$ representing the centre and $\rho=e^{-\alpha}$ as the radius.

So far we have talked about circles in the plane. We shall
now look at the treatment of circles at general positions and orientations
in space. Firstly we note that since we can think of $C$
as the wedge product of the representations of three
points on the circle, it follows that the \emph{plane}
which the circle lies in is $C \wedge n$.

In deriving equation~\ref{circle} for the circle in the plane we
used the pseudoscalar for the 4d space. It therefore seems plausible that
when we move to general circles in 3d the r\^ole of this
pseudoscalar will be taken by the plane in which the circle lies.

We firstly define the \emph{unit} plane $I_c$ by
%
\begin{equation}
  I_c = \frac{n\wedge C}{\sqrt{(-[n\wedge C]^2)}}
\end{equation}
%
since $(n\wedge C)$ always squares to give a negative
scalar. In future we will find
that it is convenient to always take the line, plane, circle,
sphere etc. of unit magnitude.
The dual of the circle $C$ is then given by the
analogous equation to equation~\ref{circle} where we
assume that given $C$, which can be `unit', we form $C^*$
and then normalise such that $C^*\dt n= -1$. That is
%
\begin{equation}
   C^* \equiv C I_c= B - \half\rho^2 n
   \label{circle2}
   \end{equation}
%
where $B$ is again the centre of the circle, $\rho$ is the radius.
Note that the `dual' in this case is with respect the `unit' plane
in which the circle lies. The proof of this is in most respects
identical to the previous proof for the plane but, where before we
used $I_4$, we now use the unit plane. This plane is proportional to
$e_1\wedge e_2\wedge \bar{n}\wedge n$, and is indeed identical to $I_4$.
We can then rotate, dilate and translate as before and note that
$R(C_o I_c)\tilde{R}=RC_o\tilde{R}RI_c\tilde{R}$ and hence our new
dual has the form given in equation~\ref{circle2}. The dual is formed by
taking the product of the transformed circle with the transformed
plane it lies in.  We also note that we can find the radius of
this general circle very simply by squaring $C^*$
%
\begin{eqnarray}
\left(C^*\right)^2 &  = &  (B - \half\rho^2n)^2 \nn \\
                  & = &  -\rho^2B\dt n = \rho^2
\end{eqnarray}
%
using the facts that $B^2=0$, $n^2=0$ and $B\dt n = -1$. From this
it then follows that $B = C^* + \frac{1}{2}(C^*)^2n$. To
summarise, from the vector form of any general circle, we can
easily obtain the centre and radius as follows
%
\begin{eqnarray}
    \left(C^*\right)^2 &  = &      \rho^2 \\
    B & = &  C^* \left[ 1 + \frac{1}{2}C^*n \right]
 \end{eqnarray}
 %
Note here that the above relations assume $C^*$  is normalised
such that $C^*\dt n=-1$ since $C^*\dt n = B\dt n = -1$, as we
assume $B$ is a normalised null vector.

While the above formulation is indeed useful, we will now see
that there is a far more elegant way of finding the
centre of a circle in 3d. The centre of a circle, $C$,  is also given 
by simply reflecting the point at infinity, $n$, in the
circle, i.e.
%
\begin{equation}
CnC
\end{equation}

To prove that $CnC$ gives the centre we can return to our circle
centre the origin, radius one. We saw earlier in this section that we can
write this circle as  $C= 2e_1e_2\bar{e}$ and, again, we can
define a \emph{unit circle}, $\hat{C}$, as $\hat{C}= e_1e_2\bar{e}$. 
Thus we see that
 %
 \begin{eqnarray}
 \hat{C}n\hat{C} &  = &  e_1e_2\bar{e}ne_1e_2\bar{e} \nn \\
        & = &  -e_2\bar{e}n e_2 \bar{e} \nn \\
        & = &  -\bar{e}n\bar{e}  \nn \\
        & = & \bar{e}(e + \bar{e})\bar{e} = e - \bar{e} \nn
        \\
        & = &  \bar{n}
\end{eqnarray}
%
which is indeed the origin and therefore the centre of
the circle. Having proved that the result holds for this
simple case, we can now rotate, dilate and translate our
circle to give any other circle and the result will still
hold. Suppose we apply a rotor, $R$, which is a
composition of rotors which rotate, dilate and translate,
to take our circle, $C$, to any other general circle,
$C'=R C \tilde{R}$. Then we see that
%
\begin{equation}
C'nC' \propto RC\tilde{R}(Rn\tilde{R}) RC\tilde{R} \propto
R(CnC)\tilde{R} \label{eqCnC}
\end{equation}
%
since we have seen previously that $Rn\tilde{R} \propto n$ for $R$
composed of rotations, translations and dilations.
Equation~\ref{eqCnC} thus tells us that the rotated origin,
$R(CnC)\tilde{R}$, is indeed given by $C'nC'$. This type of simple
proof, ie proving a result for some simple case at the origin and
generalising via rotations, translations and dilations, is a nice
feature of the conformal framework. There will many other examples
of the $XaX$ formulation producing something interesting in
subsequent sections.

In many physical systems we find that GA is very useful in that
it exposes how {\em observables} usually arise by sandwiching a
`fiducial' object between a multivector and its inverse or reverse.
For example, in quantum mechanics, we get the spin current in 3d
from a Pauli spinor, $\phi$, via $s = \phi e_3 \tilde{\phi}$. We see then
that this example of sandwiching $n$ between a circle and its
reverse (the reverse of a circle is itself up to a scale factor)
is an example of this principle at work in the conformal setting.

\subsection{Lines as trivectors}

 Next we consider the circles passing through
the point at infinity which we have already seen are
lines. Again we begin by asking ourselves what the dual
of a line, $L$, actually encodes. We suspect the answer to
this questions will provide relevant information.  
As with circles, let us begin by considering lines in the plane.
As a starting point we take the line $L$ parallel to the $x$-axis and
distance $d$ from the $x$-axis.
%, as shown in
%figure~\ref{linedual}.
%
%\begin{figure}
%\centerline{
%\includegraphics[width=.3\textwidth]{linedual.eps}
%} \caption[]{Simple line in the plane, parallel to the
%$x$-axis and distance $d$ from it} \label{linedual}
%\end{figure}
%
This line is given by wedging together any two points on
the line, say $A = F(de_2)$, $B = F(e_1 + de_2)$, and $n$;
%
\begin{eqnarray}
L & = & F(de_2)\wedge F(e_1 + de_2)\wedge n \nn \\
  & = & \frac{1}{4}(d^2n+2de_2-\bar{n})\wedge
  \left(\sqrt{(1+d^2)}n+2(e_1+de_2-\bar{n}\right)\wedge n \nn \\
  & = & -\half e_1\wedge n\wedge \bar{n} - d e_1\wedge e_2\wedge
  n
\end{eqnarray}
%
If we then take the dual of $L$, again with respect to
the pseudoscalar, $I_4$, for our 2d conformal space, we
see that
%
\begin{eqnarray}
L^* & = & LI_4 = -(\half e_1E + de_1 e_2n)e_1e_2e\bar{e} \nn \\
    & = & e_2 + dn = e_1I_2 + dn
\end{eqnarray}
%
using the identities, $e_1E=Ee_1$, $e_2E=Ee_2$,
$ne\bar{e}=n$ and $Ee\bar{e}=-2$. Thus, we see that the
dual of the line in the plane gives the dual, with
respect to the pseudoscalar in Euclidean space,
$I_2=e_1e_2$, of the Euclidean unit vector in the
direction of the line plus $dn$ where $d$ is the
perpendicular distance of the line from the origin. This
holds for any $d$, and if we apply rotors to rotate our
line (noting again that $RnR=n$) we see that we can
extend our proof to any line in the plane, therefore
giving
%
\begin{equation}
L^* = \hat{m}I_2 + dn
\end{equation}
%
where $\hat{m}$ is the unit vector in the direction of
the line and $d$ is the perpendicular distance of the
line from the origin.

Recalling that $n\dt\bar{n}=2$, we can extract $d$ easily
from $L^*$ as follows
%
\begin{equation}
 d = \half L^*\dt \bar{n}
\end{equation}
%
Similarly we can then extract $\hat{m}$ as
%
\begin{equation}
   \hat{m} = -[L^* - \half (L^*\dt\bar{n})n]I_2
\end{equation}
%

As with circles, it should be straightforward to
extend this work on lines from the plane into space.
However, unlike the circle case, we do not have an
obviously specified plane with which to define a 4d
pseudoscalar, so instead we look at the dual of the line
with respect to the pseudoscalar for the whole 5d space.
Clearly we multiply a 3-vector with a 5-vector and get a
2-vector. Once again, consider the same line that we
considered above.
%, depicted in figure~\ref{linedual} --
%again we have $L = -\half e_1\wedge n\wedge \bar{n} - d
%e_1\wedge e_2\wedge n$. Now 
We look at the dual of this line with $I\equiv I_5$
%
\begin{eqnarray}
L^* & = & -(\half e_1\wedge n\wedge \bar{n})I - d(e_1 e_2 n)I
\nn \\
    & = &  e_2e_3 - d(e_1\wedge e_2)In \nn \\
    & = &  e_1I_3 + [((de_2)\wedge e_1))I]n
\end{eqnarray}
%
since $In=nI$. Clearly the first term gives us the unit
vector in the direction of the line and the second term
gives us its moment. As this holds for any $d$, if we
rotate our line we can produce any line in space and we
see that the above can therefore be generalised to hold
for any such line. More specifically we write the dual
of the line $L$ as
%
\[ L^* = \hat{m} I_3 + [(a \wedge \hat{m})I_3]n  \]
%
where $\hat{m}$ denotes the unit vector (in 3d) in the
direction of the line and $a$ is any 3d point on the
line. This is analogous to writing the line in terms of
Pl\"ucker coordinates, where 3 of the coordinates give
the line's direction and the other 3 give its moment
about the origin.

\section{4-Vectors}

We have already seen that 4-vectors in the conformal
setting can represent both spheres and planes. Having
seen how we interpret the duals for circles and lines it
will now be easy to extend these arguments to spheres and
planes.

\subsection{Spheres as 4-vectors}

Given any 4 points whose 5d representations are
$P,Q,R,S$, the sphere through those points is given by
the 4-vector $\Sigma$
%
\[  \Sigma = P\wedge Q \wedge R\wedge S   \]
%
We know that $X \wedge \Sigma = 0 $ for any $X$ lying on
the sphere.  This can be rewritten as
%
\[ X \dt (\Sigma I) =0  \;\;\; \implies \;\;\; X\dt \Sigma^{*} = 0  \]
%
where $\Sigma^{*}= \Sigma I$ is the dual to
$\Sigma$ and, hence, is a vector. We can now follow precisely the
same workings given for the circle to show that the dual
representation of the sphere naturally encodes the centre
and the radius.

If $X$ is a point on a sphere and $C$ is its centre we
know that we can write
%
\[  X\dt C = -\half(x-c)^2 \equiv -\half\rho^2  \]
%
where $\rho$ is the radius of the sphere. For a
normalised point $X$ ($X\dt n = -1$) this therefore
implies that
%
\[ X\dt (C - \half\rho^2n) =0 \]
%
Comparing this with $X\dt \Sigma^*$ we see that provided
we normalise $\Sigma^*$ after taking the dual (such that
$\Sigma^*\dt n = -1$), then we will find
%
\begin{equation}
   \Sigma^* = C - \half\rho^2 n
   \end{equation}
%
As before, the dual  vector $\Sigma^*$ encodes the
centre and radius of the sphere. Whether one wishes to
use $\Sigma$ or $\Sigma^*$ depends on whether it is most
useful to specify the sphere by 4 points lying on it or
by its centre and radius. Given a $\Sigma^*$ (via taking
the normalised form of the dual of $\Sigma = P\wedge Q\wedge
R\wedge S $) we can immediately get the radius and centre
in the manner outlined earlier for the circle
%
\begin{eqnarray}
    \left(\Sigma^*\right)^2 &  = &      \rho^2 \\
    C & = &  \Sigma^* \left[ 1 + \frac{1}{2}\Sigma^*n \right]
 \end{eqnarray}
 \label{centre_radius}
 %
As we saw with the case of circles, there is also a more
elegant means of extracting the centre of a sphere given
$\Sigma$ or $S = \Sigma^*$. The method is to reflect the
point at infinity, $n$, in the sphere, so that the
centre, $C$, is given by
%
\begin{equation}
  C = SnS = \Sigma n \Sigma
  \end{equation}
%
To show this we consider the sphere of radius 1 centred
on the origin, $\Sigma_o = F(e_1)\wedge F(e_2)\wedge
F(e_3)\wedge F(-e_1)$ which we can expand to give
$2e_1e_2e_3\bar{e}$, so that the dual, $S$, is given by
the particularly concise expression $2e$; we therefore
take the 'unit' sphere to be $e$. We therefore see that
%
\begin{equation}
  SnS = ene = \bar{n}
\end{equation}
%
which is indeed the origin, which is the centre of the
sphere in this case. Now if we rotate, dilate, and
translate our sphere at the origin to any other sphere in
space, $S'$ say, then, reflecting $n$ in our new sphere
gives
%
\begin{eqnarray}
  S'nS'&  = &
  (RS\tilde{R})n(RS\tilde{R})=(RS\tilde{R})Rn\tilde{R}(RS\tilde{R})\nn
  \\
  & = & R[SnS]\tilde{R}
\end{eqnarray}
%
since $R[SnS]\tilde{R}$ is the transformed origin, i.e.
the new centre of the sphere, we see that the formula
$S'nS'$ does indeed give the centre.




\subsection{Planes as 4-vectors}
\label{sec:planes}

A plane, $\Phi$, passing through the 3 points whose 5d
representations are $P,Q,R$, is given by
%
\[ \Phi = P\wedge Q \wedge R \wedge n  \]
%
The physical quantities that we might want to extract
from such a 4-vector are clearly the normal to the plane
and the perpendicular distance of the plane from the
origin. We now investigate how the dual form of the plane
helps us to do this. Once again we start by considering
the plane $z=d$ which is parallel to the $x$-$y$ plane
and distance $d$ from it. We can represent $\Phi$ by
%
\begin{eqnarray}
\Phi &  =  &  F(de_3)\wedge F(e_1 + de_3) \wedge
F(e_2+de_3)\wedge n \nn \\
     &  =  & \frac{1}{8} \left\{ (2de_3-\bar{n})\wedge
     (2[e_1+de_3]-\bar{n})\wedge (2[e_2+de_3]-\bar{n})\wedge
     n\right\}  \nn \\
     &  =  &  de_1\wedge e_2 \wedge e_3\wedge n - \half e_1\wedge
     e_2\wedge \bar{n}\wedge n \nn \\
     &  =  & de_1\wedge e_2 \wedge e_3\wedge n - e_1\wedge
     e_2\wedge e\wedge \bar{e}
     \label{planeeq1}
\end{eqnarray}
%
where the second line in equation~\ref{planeeq1} follows
because wedging with $n$ removes any term containing $n$.
Then it is simple to show that the dual of $\Phi$ is
given by
%
\begin{equation}
\Phi^* = \Phi I=  dn + e_3
\end{equation}
%
This holds for any $d$. If we rotate this plane via a
rotor we know that the proof must hold and see therefore
that since $Rn\tilde{R}=n$ and $Re_3\tilde{R}=\hat{n}$,
where $\hat{n}$ is the new rotated normal to the plane,
the general equation for the dual of the plane is given
by
%
\begin{equation}
\Phi^* = dn + \hat{n}
\end{equation}
%
Note that the above assumes that we have normalised the
dual such that $[\Phi^*]^2=1$; we can ensure this by
normalising the plane such that $\Phi^2 = 1$. Therefore,
given 3 points on the plane, $P,Q,R$, we form the
normalised plane $\Phi$ and its dual $\Phi^*$, and we can
then extract $\hat{n}$ and $d$ as follows
%
\begin{eqnarray}
 d  & = &  \half \Phi^*\dt \bar{n} \\
   \hat{n} &  = &  \Phi^* - \half (\Phi^*\dt\bar{n})n
\end{eqnarray}
%
Whether we use the 4-vector form or the  \emph{normal-distance} 
form of the plane will depend upon the
problem we are solving.


\section{Intersections}

In this section we outline the various ways of intersecting to
objects within the conformal model,

\subsection{Spheres with spheres or planes}

Let us consider the intersection of two spheres, $\Sigma_1$ and
$\Sigma_2$ where they may intersect in a circle, at a point or not
at all. Suppose we take the formula for the meet (where now we use
$*$ to indicate multiplication by the pseudoscalar $I_n$):
%
\begin{equation}
C = \Sigma_1 \vee \Sigma_2 = \left[\left< \Sigma_1 \Sigma_2
\right>_{2n-r-s}\right]^*
\end{equation}
%
$2n-r-s = 10-4-4=2$, so that the dual quantity will have
grade $5-2=3$ --- generally, this will give the trivector
representing the circle of intersection. We can tell
whether we have a circle, a point intersection or no
intersection according to whether
%
\begin{equation}
C^2 >0 \;\;\;\; C^2 = 0 \;\;\;\mbox{or}\;\;\; C^2<0
\end{equation}
%
In the case of $C^2>0$ we can extract the centre and radius
according to equation~\ref{centre_radius}. If we also attempt to
extract the centre and radius via these same formul\ae\ from $C$
where $C^2=0$, we will find that the circle will have zero radius
and its centre will be the point of tangency of the two spheres.
Similarly, attempting to extract the radius and centre from $C$ in
the case $C^2<0$ (i.e. no intersection) leads to an imaginary
radius and a centre which lies on the shortest line joining the
surfaces of the spheres (i.e. that joining the centres). If the
two spheres have the same radii, it is the midway point on this
line.

The above all follows through if instead of having a
second sphere, $\Sigma_2$, we have a plane, $\Phi_2$ -- we again get a
trivector for our intersection object via the meet and the sign of the square
of this trivector tells us whether the two objects are tangent, intersect in a
circle or do not intersect at all.

Although we will not go into this here, it is interesting
to note that in the case of the non-intersecting objects,
we can also extract information such as the perpendicular
vector between the surfaces from the product $W_rW_s$.

%
%\subsection{The method of projectors}
%
%Some intersections discussed below, namely line-sphere and line-circle give, in general, two points of
%intersection. In these cases we find that the meet generates a new class of object, a
%bivector representing a pair of points; the points $A$ and $B$ are simply represented
%by $A \wedge B$. It is simple to show that the null-space of this has the 
%required form since
%\[
%(A \wedge B) \wedge X = A \wedge (B \wedge X) =  0
%\]
%is only true in general if $B = X$ and, by symmetry, if $A = X$. 
%
%Extracting $A$ and $B$ from $A \wedge B$ requires a little care and may be
%preformed using the method of projectors.  Let our bivector be $T = A \wedge
%B$. Firstly consider the effect of taking the inner product of $T$ with $n$ 
%
%%\textbf{FIXME}: Should I explain this more
%%fully?
%\[
%T \cdot n = -n \cdot T = -(n\cdot A)B + (n\cdot B)A = B-A
%\]
%which is true as the identity $n \cdot F(x) = -1$ can be shown by 
%direct substitution. Now form
%\[
%F = \frac{1}{\beta}A \wedge B
%\]
%where $\beta = T^2$ and hence $F^2=1$ (we shall consider the special 
%case $\beta =0$ later). We now define two \emph{projector} operators in terms of $F$
%\[
%P = \frac{1}{2}(1 + F)
%\]
%\[
%\tilde{P} = \frac{1}{2}(1 - F)
%\]
%
%As suggested by the operator naming, $\tilde{P}$ is the reverse of $P$ (the reverse of a 
%bivector $F$ is simply $-F$). These operators also have the following interesting 
%property
%\begin{eqnarray*}
%P^2 & = & \frac{1}{4} (1+F)(1+F) \\
%    & = & \frac{1}{4} (1 + 2F + F^2) \\
%    & = & \frac{1}{4} (2 + 2F) \\
%    & = & \frac{1}{2} (1 + F) = P
%\end{eqnarray*}
%and similarly $\tilde{P}^2 = \tilde{P}$. It is also trivial to show that $P\tilde{P}=0$ and
%that
%\[
%PA = \tilde{P}B = 0
%\]
%\[
%PB = B, \  \tilde{P}A = A
%\]
%We can now extract $A$ and $B$ from $T$
%\begin{equation}
%P \left[ T\cdot n \right] = P (B - A) = PB-PA = B\label{eqn:extractB}
%\end{equation}
%\begin{equation}
%\tilde{P} \left[ T \cdot n \right] = \tilde{P} (B-A) = \tilde{P}B - \tilde{P}A = -A
%\label{eqn:extractA}
%\end{equation}
%
%Remember that this approach was only valid if $\beta^2 \ne 0$. Let's consider the
%form of $\beta_2$ a little more 
%\begin{eqnarray*}
%\beta & = & (A \wedge B)(A \wedge B) \\
%      & = & \frac{1}{4}(AB - BA)(AB - BA) \\
%      & = & \frac{1}{4}(ABAB - ABBA - BAAB + BABA) \\
%      & = & 0 \quad\mbox{ iff } ABAB + BABA = ABBA - BAAB
%\end{eqnarray*}
%The condition for $\beta = 0$ can be simplified further
%\begin{eqnarray*}
%ABAB + BABA & = & ABBA - BAAB \\
%            & = & AB^2A - BA^2B \\
%	    & = & 0 - 0 = 0 \\
%\Rightarrow ABAB & = & -BABA 
%\end{eqnarray*}
%
%This condition is only satisfied in general if $A = B$ and hence 
%if there is only one point of intersection, i.e.\ the line
%is tangential to the sphere. A similar type of analysis can be used to
%show that there is no intersection if $\beta^2 < 0$.
%We thus have a general method for factorising $A \wedge B$.

\subsection{Spheres with circles or lines}

Let us now intersect a sphere $\Sigma_1$ (4-blade) with a
circle $C_2$ (3-blade). According to our meet formul\ae
our intersection is a 2-blade, $B$, given by
%
\begin{equation}
B = \Sigma_1 \vee C_2 = \left[\left< \Sigma_1 C_2
\right>_{2n-r-s}\right]^*
\end{equation}
%
where $2n-r-s=10-4-3=3$, so that the dual object has
grade 2. We have already seen that these 2-blades
represent 2 points --- precisely as we would expect, since
an intersecting sphere and circle will do so at 2 points.
Now, again, we look at the sign of the resulting 2-blade,
$B$, and we will find that there are two, zero or one
point of intersection according to
%
\[  B^2>0 \;\;\;\; B^2=0 \;\;\;\mbox{or}\;\;\;\; B^2<0.
\]
%
We have seen earlier that given a bivector $B$, such that
$B^2>0$, of the above form, we can extract the two points
of intersection via the projectors given in
equations~\ref{eqn:extractAB}. If $B^2=0$ we cannot form
the projector, but it is trivial to find the representation of the point of
intersection, $X$, in this case using the following
%
\[  X = BnB  \]
%
i.e. for a 2-blade of the form $W = P\wedge Q$, reflecting
$n$ in $W$, $WnW$, would give us the midpoint of the line
joining $P$ and $Q$ --- for our case where $B^2=0$, the
construction $BnB$ will therefore give us a representation of the point of
intersection. These results can easily be shown by
considering simple cases at the origin and then extending
the proof via rotors as previously.

Precisely the same working holds if we replace our
circles above with lines --- the meet again gives a
2-vector whose square tells us whether there are 2, 1 or
no intersections, and from which the intersection points
can be obtained easily. We will return to the
intersections of lines with spheres when we later
consider reflections of lines in spheres.

Again, in the non-intersecting case the product $W_rW_s$
can provide us with information about the vectors between
the surfaces.




\subsection{Planes with planes, circles and lines }

Consider two planes $\Phi_1$ and $\Phi_2$; taking the
meet gives
%
\begin{equation}
L = \Phi_1 \vee \Phi_2 = \left[\left< \Phi_1 \Phi_2
\right>_{2n-r-s}\right]^*
\end{equation}
%
where $2n-r-s=10-4-4=2$, so that the dual object has
grade 3 --- as we would expect, if the planes intersect to
give a line. We are able to tell whether the planes
intersect by looking at the sign of $L^2$ --- if $L^2=0$
we know that the planes are parallel and do not
intersect, if $L^2>0$, the planes intersect in the line
$L$.

Now consider a plane $\Phi_1$ and a circle $C_2$; we take
the meet of these two objects to give
%
\begin{equation}
B = \Phi_1 \vee C_2 = \left[\left< \Phi_1 C_2
\right>_{2n-r-s}\right]^*
\end{equation}
%
where $2n-r-s=10-4-3=3$, so that the dual object has
grade 2 --- the plane and the circle intersect in a
maximum of two points, and the 2-blade, $B$ encodes these
two points as with the sphere-circle intersection. Once
again we can assert that there are 2,1 or 0 intersections
according to whether $B^2>0,B^2=0,B^2<0$. In the case of
two intersections, the points are extracted from $B$ by
projectors as before, and in the case of tangency, the
one point of contact is obtained by taking $BnB$.

It is worth thinking about what happens when the circle
$C_2$ lies in the plane $\Phi_1$ so that the intersection is
$C_2$ itself. As one might expect, in this case there \emph{is
no} grade 3 part of $\Phi_1C_2$. In the case of the $z=0$ plane and
the unit circle lying in the plane and centred on the origin this can
easily be confirmed:

\[
\Phi_1 = F(e_1) \wedge F(e_2) \wedge F(-e1) \wedge n \propto e_1e_2e\bar{e}
\]
\[
C_2 = F(e_1) \wedge F(e_2) \wedge F(-e1) \propto e_1e_2\bar{e}
\]
thus
\[
\Phi_1C_2 \propto e_1e_2e\bar{e}e_1e_2\bar{e} = e \mbox{ hence }
\left<\Phi_1C_2\right>_3 = 0
\]

We can also note that, in this case, the dual of $\Phi_1C_2$ with respect to
$e_1e_2e\bar{e}$ is indeed $C_2$.

If we now replace our circle by a line, $L_2$, it is
clear that the meet will still give us a 2-blade, $B$;
but we know that the line and the plane intersect in at
most one position, so should we not be looking for a
vector rather than a 2-blade? The answer is that if the
plane and the line intersect, and the meet gives us $B$,
then $B$ is always of the form
%
\[  B = X\wedge n  \]
%
where $X$ is the representation of the point of intersection. This can be
proved easily by again considering a simple case at the
origin. If $B^2>0$ the line and plane intersect in a
point, if $B^2=0$ the line and plane do not intersect and
if $B=0$ the line lies in the plane. If there is one
point of intersection so that $B$ is of the above form,
we can extract the three-dimensional point of intersection,
$x=x^ie_i,\;i=1,2,3$ (and hence $X$), by simply equating
$x^i$ to the coefficient of the $e_i\wedge n$ term or by
using the following expansion
%
\begin{equation}
 x = (B\wedge \bar{n})\cdot E
 \end{equation}
%
where $E=n\wedge \bar{n}$ as given earlier.


\subsection{Circles with circles and lines }

Consider two circles, $C_1$ and $C_2$, taking their meet
gives
%
\begin{equation}
X = C_1 \vee C_2 = \left[\left< C_1 C_2
\right>_{2n-r-s}\right]^*
\end{equation}
%
where $2n-r-s=10-3-3=4$, so that the dual object has
grade 1. We know, however, that the intersection of two
circles has at most 2 intersections (only possible if
they lie in the same plane), so how do we get two
intersections from our grade 1 object? In fact we find
that the following is true
%
\begin{eqnarray}
C_1 \vee C_2  &  =  &  X \;\;\mbox{where}\;\;
X^2=0\;\;\mbox{if circles have one intersection} \nn \\
C_1 \vee C_2  &  =  &  X \;\;\mbox{where}\;\;
X^2\ne 0\;\;\mbox{if circles have no intersection} \nn \\
C_1 \vee C_2  &  =  &  0 \;\; \;\;\mbox{if circles have
two intersections}.
\end{eqnarray}
%
In the case where the meet gives zero and we know there
are two intersections, these can easily be found by
intersecting the plane of one of the circles with the
other circle, i.e.
%
\begin{equation}
B = C_1 \vee (C_2\wedge n) =  \left[\left< C_1 (C_2\wedge n)
\right>_{2n-r-s}\right]^*
\end{equation}
%
where $2n-r-s=10-3-4=3$, so that the dual object has
grade 2, and the two points of intersection can be
extracted from the 2-blade $B$ using
equation~\ref{eqn:extractAB}.

If we now replace $C_2$ by a line $L_2$ we see that we
again get a grade 1 object when we take the meet, and the
situation above is exactly replicated, i.e.
%
\begin{eqnarray}
C_1 \vee L_2  &  =  &  X \;\;\mbox{where}\;\;
X^2=0\;\;\mbox{if circle and line have one intersection} \nn \\
C_1 \vee L_2  &  =  &  X \;\;\mbox{where}\;\;
X^2\ne 0\;\;\mbox{if circle and line have no intersection} \nn \\
C_1 \vee L_2  &  =  &  0 \;\; \;\;\mbox{if circle and
line have two intersections}
\end{eqnarray}
%
As before, in the case where the meet gives zero the two
intersections can easily be found by intersecting the
plane of one of the circles with the line. It is also
interesting to note here that in the case where the
circle and the line do not intersect, with the meet
giving a vector, $X$, which is not null, the sign of
$X^2$ tells us whether the line passes through the circle
($X^2<0$) or does not pass through the circle ($X^2>0$)
-- such simple checks can often be useful in graphics
applications.


Given that we have had a little difficulty with circles
intersecting circles, we might expect some slight
difficulties with lines and lines. It turns out that many
interesting constructions emerge when we start to
consider the intersections between two lines, these will
be discussed in the following section.


\subsection{Lines with lines }

Let us consider two lines, $L_1$ and $L_2$. Taking the
meet of these two lines gives
%
\begin{equation}
X = L_1 \vee L_2 = \left[\left< L_1 L_2
\right>_{2n-r-s}\right]^*
\end{equation}
%
where $2n-r-s=10-3-3=4$, so that the dual object has
grade 1. We might expect that if the lines intersect at a
point, the meet, $X$, will give this intersection point
-- however, this is not the case. We find that the
following is true
%
\begin{eqnarray}
L_1 \vee L_2 & = & 0 \;\;\; \mbox{if the lines intersect}
\nn \\
L_1 \vee L_2 & \propto & n \;\;\; \mbox{if the lines do
not intersect}
\end{eqnarray}
%
We therefore have a simple of way for checking for
intersecting lines, but if the meet gives us zero so that
we know there is an intersection point, we can no longer
find this point in a fully co-variant way by intersecting
one line with the a plane defined by the other line,
since such a plane is not uniquely defined. We could, in
practise, intersect one line with the plane formed by the
other line and the origin, but then if the other line
passes through the origin, this will not work, and we are
left with a non-co-variant procedure and one which entails
us forming conditionals for a number of cases. We would
instead like to look for a method which works co-variant
-- such a method exists, and in the process of describing
it, we see a number of other useful constructions.

Again take our arbitrary lines, $L_1$ and $L_2$ (assume
they are normalised, such that $L_1^2=L_2^2=1$). Suppose
we reflect line $L_1$ in line $L_2$ --- this statement is
not well-defined in a conventional sense, but in
GA, we have seen that reflection of an
object in another object is indeed well defined and is
brought about by sandwiching the object to be reflected
between the object that it is being reflected in. So, our
reflected line, $L_1'$ is given by
%
\[  L_1' = L_2L_1L_2  \]
%
The operation of reflection is grade preserving if we are
dealing with blades, and therefore we know that we get
another line. This construction will work for any two
lines, but let us now suppose that our lines intersect at
a point, we can then expect that the reflected line
$L_1'$ will be the line formed by the intersection point,
represented by $P$, and any point on $L_1$ reflected in $L_2$. This is
indeed what $L_1'$ is. Now, however, it becomes possible
very easily to form the line which is perpendicular to
$L_2$, passing through the intersection point, $P$, and
in the plane defined by $L_1$ and $L_2$ via the following
%
\begin{equation}
 L_1'' =  L_1 - L_2L_1L_2
\end{equation}
%
%-- this is illustrated in figure~\ref{intersect_lines}
%
%\begin{figure}
%\centerline{
%\includegraphics[width=.4\textwidth]{c:/jl/siggraph/2002/intersect_lines}
%} \caption[]{The rotation of intersecting lines to
%produce two lines intersecting at right angles, via the
%construction $L_1-L_2L_1L_2$.} \label{intersect_lines}
%\end{figure}
%
\begin{figure}
\centerline{
\includegraphics[width=.4\textwidth]{lines}
} \caption{The rotation of intersecting lines to
produce two lines intersecting at right angles, via the
construction $L_1-L_2L_1L_2$.} \label{intersect_lines}
\end{figure}
Clearly what we are doing here is rotating one line in
the plane defined by the two lines,  to be perpendicular
to the other line. We will return to this rotor
description later.  Now we have two perpendicular lines
which intersect, we can find the point of intersection
relatively easily. Take any arbitrary point representation $X$ and
reflect it in $L_1''$ (assuming again that we have
normalised $L_1''$) via $X' = L_1''XL_1''$, then take the
midpoint of $X$ and $X'$, we know that this must lie on
the line $L1'$
%
\[ X' = L_1''XL_1'' \;\;\;\; X'' = \half (X + X')   \]
%
(recall that $X''$ will be our real midpoint plus some
multiple of $n$). Now we reflect $X''$ in $L_2$ to give $X'''$ and again
take the midpoint --- the midpoint must now give us the
intersection point $P$ plus some multiple of $n$;
%
\[ X''' = L_2X''L_2 \;\;\;\;  P' = \half (X''+X''')   \]
%
we then extract the null vector corresponding to the representation of our
intersection point $P$ via
%
\[ P = \frac{-(P'nP')}{2(P'\cdot n)^2}   \]
%
This construction is independent of $X$ and is a
beautiful illustration of the ability to manipulate
objects using geometric algebra. Although it appears
involved, it can be programmed up very easily ($\bar{n}$
can be used as the point $X$) and is an entirely
co-variant way of intersecting two lines. Of course, in
practise, one can also check to see if there is at least
one line that does not pass through the origin
($\bar{n}\wedge L=0$ if $L$ passes through the origin), if
there is, say this is $L_1$, we can then form the plane
$\bar{n}\wedge L_1$ and intersect this with $L_2$, if both
lines pass through the origin then the intersection point
is the origin. Note that we can use precisely the same
type of argument to extract the plane formed by two
intersecting lines. For example, take any arbitrary point
$X$, reflect it in the line $L_1$ via $L_1 X L_1$, so
that the midpoint of this line is given by $P$ where $P$
is given (up to some additional multiple of $n$) by
%
\[  P = \half (X + L_1 X L_1)   \]
%
$P$ must clearly lie on $L_1$, thus the plane formed by
the two lines must be given by
%
\begin{equation}
 L_2\wedge P = L_2 \wedge \half(X + L_1X L_1)
\end{equation}
%
Again, $\bar{n}$ can be used for our point $X$ in real
computations. The derivations given here are entirely
co-variant and the same constructions will intersect `lines' in different
geometries, we will illustrate this towards the end of the report by
considering three-dimensional hyperbolic geometry.

Recall that
previously we found the reflection of $L_1$ in $L_2$ via
$L_2L_1L_2$, now note that we can rewrite this as
%
\begin{equation}
 L_2L_1L_2 = (L_2L_1)L_1\widetilde{(L_2L_1)} = RL_1\tilde{R}
\end{equation}
%
since $\widetilde{(L_2L_1)} = \tilde{L}_1\tilde{L}_2=L_1L_2$.
Thus we see that the quantity $L_2L_1$ acts as a rotor
(if the lines are normalised) which rotates through twice
the angle between the lines about an axis through the
intersection point and perpendicular to the plane
containing the lines. In fact, when we write this
reflection as a rotation, there is nothing to insist that
our lines must intersect. If $L_1$ and $L_2$ do not
intersect, then $L_2L_1L_2$ will still perform a {\em
reflection of $L_1$ in $L_2$}, but we are now able to
interpret exactly what this reflection means for
non-intersecting lines if we regard the operator as a
rotor. It turns out that the rotor $L_2L_1$ is the
product of a rotation rotor and a translation rotor --
the rotation is in the plane normal to the common
perpendicular of the lines and the translation is along
the common perpendicular so that one line is taken onto
and then through the other. We can show that it is
possible to write the product $L_2L_1$ as
%
\begin{equation}
L_2L_1 = (\cos{\theta} + \hat{B}\sin{\theta})(1 + dn)
\end{equation}
%
where $d$ is the three-dimensions vector representing the length and
direction of the common perpendicular, $\hat{B} =
\hat{d}I_3$ (with $\hat{d}=d/\sqrt(d^2)$) and  $\theta$
is the angle between the lines as measured when
translated to lie in the same plane along $d$. We see
that the above is a combination of a rotation rotor and a
translation rotor, we will see later that rotors which
take one object (of the same grade) into another object
are often formed in the way we have outlined here for
lines.


