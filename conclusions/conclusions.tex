\chapter{Conclusions and Future Work}

In this chapter we shall collate all of the findings from the previous chapters
and give them a context in relation to each other. Future applications for
the various findings will also be discussed.

\section{Review of Achievements}

In this section we briefly review the achievements and findings from each
chapter.

\subsection{Non-Euclidean geometries}

In chapter \ref{chap:noneuclid}, a framework for extending the conformal
model to deal with non-Euclidean geometries was shown, with particular
emphasis on hyperbolic geometry. It was shown that the geometry
represented by a model is entirely determined by the choice of null-vector
representation and rotors. The pure-rotation and pure-translation 
rotors for hyperbolic space were derived and it was shown that from them the usual
distance metric for hyperbolic geometry could be obtained.

Already some work using the conformal model to represent non-Euclidean geometry
has found application in cosmology\cite{GA:SIGKEY} has been done leading, potentially, to important insights on
our universe.

\subsection{Fractals}

In chapter \ref{chap:fractals} an extension to complex numbers, similar to that
of quaternions, was developed for arbitrary dimension. It was noted that, in
GA, quaternions are simply special cases of a wider variety of algebras. This
extension was used to form a dimension-agnostic formulation for the classic
complex iteration-based Julia and Mandelbrot fractal sets. In addition an
existing distance estimation formula was shown to be valid using this extension
allowing for the ray-tracing of arbitrary dimension sets.

Real-world applications of fractals are notoriously difficult to find but the
opening up of escape-time fractals to non-Euclidean geometries provides a
number of opportunities for `recreational mathematics' and the generation of
attractive images.

\subsection{Rotor exponentiation}

In chapter \ref{chap:exponential} it was hypothesised that all the rotors we
used in the conformal model could be obtained by exponentiating a corresponding
generator bivector the components of which would be geometrically meaningful.
A closed form solution for \emph{both} the exponentiation and subsequent
inverse exponentiation (modulo the identification of rotations by $2n\pi$) was
derived.

From this an algorithm for directly mapping the components of the generator
to a $4 \times 4$ matrix suitable for use in existing graphical pipelines
was developed. A matching algorithm for directly converting a matrix to
a generator, again identifying rotations of $2n\pi$, was also developed.

This particular chapter has almost limitless application. Not only is the
linear space of the bivectors mapped to the non-linear space of rigid-body
transformations but the appropriate inverse mapping was also defined. Using
this method many existing linear optimisation algorithms or interpolation
schemes could be extended to deal with rotation and translation
\emph{simultaneously}.

\subsection{GPU-based techniques}

In chapter \ref{chap:gpu} the techniques developed in chapter \ref{chap:exponential}
were implemented on the programmable portion of modern Graphics Processing
Units. Such \emph{shaders} were used to develop sample graphics algorithms which
made use of the mappings developed in this thesis. 

Specifically simple mesh deformation and collision detection examples were show.
The examples demonstrated that not only was GA a natural language for developing
such algorithms allowing one to use much geometric insight but they were also
compact enough to program so that they could be efficiently implemented in hardware.

\section{Future work}

Of all the work presented in this thesis perhaps that with the clearest scope for
future work is the work on rotor exponentiation. The ability to map rotors conveniently
into a 6d linear space allows a great deal of algorithms initially developed for
translation and points to be converted into algorithms acting on rotations. In addition
the freedom to move wherevoer one wishes within this space coupled with a well defined 
`distance' between rotors (letting the components of the bivector be those of a 6d vector and
using the normal Euclidean distance) allows one to investigate minimisation algorithms for
fitting rotations to data.

Already, unpublished work has shown promise for this approach in animation interpolation
and compression using motion capture data and further work will hopefully be fruitful in this
area.
